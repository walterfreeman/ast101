 \documentclass[12pt]{article}
\setlength\parindent{0pt}
\usepackage{amsmath}
\usepackage{lscape}
\usepackage{graphicx}
\usepackage{fullpage}
\usepackage[margin=0.8in]{geometry}
\setlength{\parskip}{4mm}
\def\LL{\left\langle}   % left angle bracket
\def\RR{\right\rangle}  % right angle bracket
\def\LP{\left(}         % left parenthesis
\def\RP{\right)}        % right parenthesis
\def\LB{\left\{}        % left curly bracket
\def\RB{\right\}}       % right curly bracket
\def\PAR#1#2{ {{\partial #1}\over{\partial #2}} }
\def\PARTWO#1#2{ {{\partial^2 #1}\over{\partial #2}^2} }
\def\PARTWOMIX#1#2#3{ {{\partial^2 #1}\over{\partial #2 \partial #3}} }
\newcommand{\BE}{\begin{displaymath}}
\newcommand{\EE}{\end{displaymath}}
\newcommand{\BNE}{\begin{equation}}
\newcommand{\ENE}{\end{equation}}
\newcommand{\BEA}{\begin{eqnarray}}
\newcommand{\EEA}{\nonumber\end{eqnarray}}
\newcommand{\EL}{\nonumber\\}
\newcommand{\la}[1]{\label{#1}}
\newcommand{\ie}{{\em i.e.\ }}
\newcommand{\eg}{{\em e.\,g.\ }}
\newcommand{\cf}{cf.\ }
\newcommand{\etc}{etc.\ }
\newcommand{\Tr}{{\rm tr}}
\newcommand{\etal}{{\it et al.}}
\newcommand{\OL}[1]{\overline{#1}\ } % overline
\newcommand{\OLL}[1]{\overline{\overline{#1}}\ } % double overline
\newcommand{\OON}{\frac{1}{N}} % "one over N"
\newcommand{\OOX}[1]{\frac{1}{#1}} % "one over X"
\pagenumbering{gobble}

\begin{document}
\Large
\centerline{\sc{Quiz 2 -- The Motion of the Zodiac (retake)}}

\begin{minipage}{0.6\textwidth}
	\Large
	Name: \underline{\hspace{3in}}
\end{minipage}
\begin{minipage}{0.4\textwidth}
	\Large
	Lab Section: M0\underline{\hspace{1in}}\\
	\small (if you want your paper back)
\end{minipage}

\normalsize

For this quiz, you may use your Homework 2 or anything you yourself have handwritten, but may not consult with anyone else or use electronic devices.

Give your paper to Walter or put it in his mailbox in room 201 when you are done.



\begin{center}
\includegraphics[width=6in]{earth-orbit-quiz-crop.pdf}

\it This is a diagram of the Earth in September, seen from above the North Pole. From this perspective, the Earth rotates on its axis counterclockwise and orbits the Sun counterclockwise.

\bigskip

The questions are on the back page.

\end{center}
\newpage
\begin{enumerate}



	\item Draw a stick figure in a location on the Earth (shown in the diagram) at midnight.

\vspace{1in}
	
	\item What constellation is just setting over the western horizon at this time?
	\vspace{1in}
	
     \item How long would your observer need to wait to see the constellation Gemini? What would need to happen?
     \vspace{1in}
     

\item Draw Earth at its position in the orbit when Pisces is behind the Sun. Then answer the following:

\begin{enumerate}
	
	\item How many months after September would this be?
	
	\vspace{1in}
	
	
	\item What time of day would Sagittarius rise?
	
	\vspace{1in}
	
	\item What constellation would be highest in the sky when this happens?
\end{enumerate}

\end{enumerate}	
	

\end{document}

