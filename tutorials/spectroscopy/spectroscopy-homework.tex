\documentclass[12pt]{article}
\setlength\parindent{0pt}
\usepackage{amsmath}
\usepackage{lscape}
\usepackage{graphicx}
\usepackage{fullpage}
\usepackage[margin=0.8in]{geometry}
\setlength{\parskip}{4mm}
\def\LL{\left\langle}   % left angle bracket
\def\RR{\right\rangle}  % right angle bracket
\def\LP{\left(}         % left parenthesis
\def\RP{\right)}        % right parenthesis
\def\LB{\left\{}        % left curly bracket
\def\RB{\right\}}       % right curly bracket
\def\PAR#1#2{ {{\partial #1}\over{\partial #2}} }
\def\PARTWO#1#2{ {{\partial^2 #1}\over{\partial #2}^2} }
\def\PARTWOMIX#1#2#3{ {{\partial^2 #1}\over{\partial #2 \partial #3}} }
\newcommand{\BE}{\begin{displaymath}}
\newcommand{\EE}{\end{displaymath}}
\newcommand{\BNE}{\begin{equation}}
\newcommand{\ENE}{\end{equation}}
\newcommand{\BEA}{\begin{eqnarray}}
\newcommand{\EEA}{\nonumber\end{eqnarray}}
\newcommand{\EL}{\nonumber\\}
\newcommand{\la}[1]{\label{#1}}
\newcommand{\ie}{{\em i.e.\ }}
\newcommand{\eg}{{\em e.\,g.\ }}
\newcommand{\cf}{cf.\ }
\newcommand{\etc}{etc.\ }
\newcommand{\Tr}{{\rm tr}}
\newcommand{\etal}{{\it et al.}}
\newcommand{\OL}[1]{\overline{#1}\ } % overline
\newcommand{\OLL}[1]{\overline{\overline{#1}}\ } % double overline
\newcommand{\OON}{\frac{1}{N}} % "one over N"
\newcommand{\OOX}[1]{\frac{1}{#1}} % "one over X"
\pagenumbering{gobble}
\begin{document}
\Large
\centerline{\sc{Homework 6 -- Spectroscopy}

\normalsize
\begin{center}
	Due Tuesday, November 9, before the beginning of class
\end{center}

\bigskip

\begin{enumerate}

	\item Suppose a type of atom -- we'll call it onondagium -- has the following energy levels:
		\begin{itemize}
			\item $n=1$: 4 eV
			\item $n=2$: 6 eV
			\item $n=3$: 7.8 eV
			\item $n=4$: 9.5 eV
			\item $n=5$: 10 eV
		\end{itemize}

		If you put a diffuse gas of onondagium in a glass tube like the ones you've seen in class and lab and run an electric current through it, what would its spectrum look like?





	\item Suppose a type of atom -- we'll call it syracusium -- has energy levels as follows:
		\begin{itemize}
			\item $n=1$: 0 eV (ground state)
			\item $n=2$: 2.1 eV
			\item $n=3$: 3 eV
			\item $n=4$: 5 eV
		\end{itemize}
		
		Suppose now that you soak a piece of paper in syracusium and illuminate it with 5 eV photons. What would you see with your eye, if anything? Are there any sorts of light that would be produced that are {\it not} visible?

	\item Forensic scientists sometimes use ``black lights'' -- lights that predominantly produce ultraviolet light -- in investigations. This UV light is not visible to our eyes, but when it shines on certain compounds 
		(such as blood), they glow with visible light. What is likely going on here, in light of the last problem?

	\item 

