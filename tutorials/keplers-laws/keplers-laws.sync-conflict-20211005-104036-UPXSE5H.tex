\documentclass[12pt]{article}
\setlength\parindent{0pt}
\usepackage{amsmath}
\usepackage{lscape}
\usepackage{graphicx}
\usepackage{fullpage}
\usepackage[margin=0.8in]{geometry}
\setlength{\parskip}{4mm}
\def\LL{\left\langle}   % left angle bracket
\def\RR{\right\rangle}  % right angle bracket
\def\LP{\left(}         % left parenthesis
\def\RP{\right)}        % right parenthesis
\def\LB{\left\{}        % left curly bracket
\def\RB{\right\}}       % right curly bracket
\def\PAR#1#2{ {{\partial #1}\over{\partial #2}} }
\def\PARTWO#1#2{ {{\partial^2 #1}\over{\partial #2}^2} }
\def\PARTWOMIX#1#2#3{ {{\partial^2 #1}\over{\partial #2 \partial #3}} }
\newcommand{\BE}{\begin{displaymath}}
\newcommand{\EE}{\end{displaymath}}
\newcommand{\BNE}{\begin{equation}}
\newcommand{\ENE}{\end{equation}}
\newcommand{\BEA}{\begin{eqnarray}}
\newcommand{\EEA}{\nonumber\end{eqnarray}}
\newcommand{\EL}{\nonumber\\}
\newcommand{\la}[1]{\label{#1}}
\newcommand{\ie}{{\em i.e.\ }}
\newcommand{\eg}{{\em e.\,g.\ }}
\newcommand{\cf}{cf.\ }
\newcommand{\etc}{etc.\ }
\newcommand{\Tr}{{\rm tr}}
\newcommand{\etal}{{\it et al.}}
\newcommand{\OL}[1]{\overline{#1}\ } % overline
\newcommand{\OLL}[1]{\overline{\overline{#1}}\ } % double overline
\newcommand{\OON}{\frac{1}{N}} % "one over N"
\newcommand{\OOX}[1]{\frac{1}{#1}} % "one over X"
\pagenumbering{gobble}
\begin{document}
\Large
\centerline{\sc{Tutorial-Exercise -- Kepler's Second and Third Laws}}

\normalsize

In this exercise, you'll explore Kepler's laws of orbital motion.

Remember that these exercises are not meant for you to do alone; you should work with others near you on them, and should raise your hand and ask questions as you have them.

Your fourth homework assignment is included on the back of this handout. You should complete it by class time on October 12 (or 7) and put it in your TA's mailbox.

\section{Kepler's Second Law}

As we saw demonstrated a bit ago, Kepler's second law says that an imaginary line between the Sun and a planet orbiting it sweeps out an equal amount of area in an equal amount of time. First, let's apply this to Earth's orbit, which is very close to a perfect circle.

Pretending for now that Earth's year is divided into twelve equal months and that its orbit is a perfect circle, here is a cartoon of Earth going around the Sun.

IMAGE

Using your pencil, shade the region that the imaginary line between Earth and the Sun sweeps out during January, and again during September.

Does Earth's orbit follow Kepler's second law? How do you know? 

\vspace{2in}

Kepler described his observations of planetary motion in terms of the ``area swept out'' by the line, but we can also think about the {\it speed} of Earth's motion.

Does the Earth move faster during January or during September? How do you know?

\newpage

Now, let's consider a different planet whose orbit is more eccentric. Here is a cartoon of its orbit, showing its position at the beginning of January and February, along with many other possibilities. 

Shade in the area that the line connecting the Sun to the planet would sweep out during January, as you did before.


Then, find {\it two other} points in the orbit during which the line connecting the planet to the Sun would sweep out approximately that same area. You would like to fill in the blanks in the statement:


\bf ``As it moves from point \underline{\hspace{0.5in}} to point \underline{\hspace{0.5in}}, the line connecting this planet to the Sun sweeps out the same area as it does during the month of January.''


	
	
\end{document}


