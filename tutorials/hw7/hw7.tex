\documentclass[12pt]{article}
\setlength\parindent{0pt}
\usepackage{amsmath}
\usepackage{lscape}
\usepackage{graphicx}
\usepackage{hyperref}
\usepackage{fullpage}
\usepackage[margin=0.8in]{geometry}
\setlength{\parskip}{4mm}
\def\LL{\left\langle}   % left angle bracket
\def\RR{\right\rangle}  % right angle bracket
\def\LP{\left(}         % left parenthesis
\def\RP{\right)}        % right parenthesis
\def\LB{\left\{}        % left curly bracket
\def\RB{\right\}}       % right curly bracket
\def\PAR#1#2{ {{\partial #1}\over{\partial #2}} }
\def\PARTWO#1#2{ {{\partial^2 #1}\over{\partial #2}^2} }
\def\PARTWOMIX#1#2#3{ {{\partial^2 #1}\over{\partial #2 \partial #3}} }
\newcommand{\BE}{\begin{displaymath}}
\newcommand{\EE}{\end{displaymath}}
\newcommand{\BNE}{\begin{equation}}
\newcommand{\ENE}{\end{equation}}
\newcommand{\BEA}{\begin{eqnarray}}
\newcommand{\EEA}{\nonumber\end{eqnarray}}
\newcommand{\EL}{\nonumber\\}
\newcommand{\la}[1]{\label{#1}}
\newcommand{\ie}{{\em i.e.\ }}
\newcommand{\eg}{{\em e.\,g.\ }}
\newcommand{\cf}{cf.\ }
\newcommand{\etc}{etc.\ }
\newcommand{\Tr}{{\rm tr}}
\newcommand{\etal}{{\it et al.}}
\newcommand{\OL}[1]{\overline{#1}\ } % overline
\newcommand{\OLL}[1]{\overline{\overline{#1}}\ } % double overline
\newcommand{\OON}{\frac{1}{N}} % "one over N"
\newcommand{\OOX}[1]{\frac{1}{#1}} % "one over X"
\pagenumbering{gobble}
\begin{document}
\Large
\centerline{\sc{Homework -- The Greenhouse Effect and Climate Change}}

\normalsize
\begin{center}
	Due Friday, December 10, before 5pm to your TA's mailbox
\end{center}

\bigskip

        Suppose that three asteroids (call them A, B, and C) are in orbit around the Sun, about 1 AU away. 

	\bigskip

	Remember that the thermal radiation from something that is thousands of degrees (like the Sun) is largely visible
	light, but the thermal radiation from something that is hundreds of degrees (like Earth) is long-wavelength infrared.
\bigskip

	Recall also that in Lab 9 you found that the temperature of a planet is set by the balance of thermal radiation from the Sun (sunlight) falling on its surface, and thermal radiation from its surface escaping into space.
\bigskip
\begin{itemize}

    \item	Asteroid A is just a bare rock in orbit around the Sun.

    \item Asteroid B is surrounded by a bubble of glass. Glass allows visible light to pass through, but partially blocks infrared light.

    \item	Asteroid C is surrounded instead by a cloud of dust. Dust partially blocks visible light, but allows infrared light to pass through.
\end{itemize}

\begin{enumerate}
    \item Rank these rocks in order of temperature. Which one would be coolest? Which one would be warmest? Which one would be in the middle?
    
    \vspace{1in}

    \item Volcanic eruptions generate large amounts of ash and sulfur dioxide droplets, which behave similarly to dust. Afte the large volcano at Krakatoa erupted in 1883, temperatures fell worldwide by half a degree.

	\bigskip

	Explain briefly why this happened, and how a volcanic eruption can have a temporary cooling effect on Earth's climate.
    \vspace{1.5in}
    \item How does the "pre-industrial" temperature of Earth compare to the things Earth's climate has done in the last million years during the cycle
	of ice ages?
    \vspace{1in}
    \item How has Earth's average temperature changed since the beginning of the Industrial Revolution (around 1850)?
    \vspace{1in}
    \item What range of climate outcomes (average temperature changes) are possible during the next century, depending on the choices we make regarding $\rm CO_2$ emissions?
    \vspace{1.5in}
    \item Compare the speed of human-caused climate change during the Industrial Revolution to the speed of climate changes that have happened during the last Ice Age. For a dramatic (and funny) illustration of this, see \url{https://xkcd.com/1732/}.
\end{enumerate}
\end{document}

