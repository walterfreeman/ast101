\documentclass[11pt]{article}
\usepackage{tocloft}
\usepackage{graphicx}
\usepackage{calc}
\usepackage{amssymb}
\usepackage{color}
\usepackage{array}
\usepackage[sc]{mathpazo}
\usepackage{url}
\usepackage[final]{pdfpages}
\usepackage{amsmath}

%\linespread{1.05}
\oddsidemargin=0pt
\evensidemargin=0pt
\textwidth=6.5in
\topmargin=0pt
\headheight=0pt
\headsep=0pt
\textheight=9in
% EXPERIMENTAL
%\parindent=0pt
%\parskip=3pt
\setlength{\parindent}{0cm}
\newcommand\secfont{\fontfamily{cmss}\selectfont}%\textwidth 5.5truein
\newcommand\pifheading[1]{{\secfont\textbf{#1}:}}
%\oddsidemargin -0.40truein
%\textheight 8.0truein
%\topmargin -0.25truein
\def\lo{
\mathrel{\raise.3ex\hbox{$<$}\mkern-14mu\lower0.6ex\hbox{$\sim$}}
}
\def\hi{
\mathrel{\raise.3ex\hbox{$>$}\mkern-14mu\lower0.6ex\hbox{$\sim$}}
}

\textwidth = 6.6 in
\textheight = 9.1 in
\oddsidemargin = -0.05 in
\evensidemargin = +0.05 in
\topmargin = -.1 in
\headheight = 0.0 in
\headsep = 0.0 in
\parskip = 0.06in
\newcommand\registered{{\ooalign{\hfil\raise .00ex\hbox{\scriptsize R}\hfil\crcr\mathhexbox20D}}}

%% Define a new 'leo' style for the package that will use a smaller font.
\makeatletter
\def\url@leostyle{%
  \@ifundefined{selectfont}{\def\UrlFont{\sf}}{\def\UrlFont{\small\ttfamily}}}
\makeatother
%% Now actually use the newly defined style.
\urlstyle{leostyle}

%\pagestyle{empty}
%\includeonly{previous,proposal_references}
%\includeonly{proposal_references}
%\includeonly{previous}

% TOC

\begin{document}
%%%%%%%%%%%%%%%%%%%%%%%%%%%%%%%%%%%%%%%%%%%%%%%%%%%%%%%%%%%%%%%%%%%%%
\begin{center}
\textbf{\Large
AST101: Our Corner of the Universe \\
\vspace*{0.1cm}
Lab 4: Parallax
}
\end{center}

\vspace*{0.5cm}

{\Large Name:}\vspace*{0.5cm}\\\hrule
%{\Large Student number (SUID):}\vspace*{0.5cm}\\\hrule
{\Large Lab section:}\vspace*{0.5cm}\\\hrule
{\Large Group Members:}\vspace*{0.5cm}\\\hrule
\vspace*{0.5cm}

%%%%%%%%%%%%%%%%%%%%%%%%%%%%%%%%%%%%%%%%%%%%%%%%%%%%%%%%%%%%%%%%%%%%%
\section{Introduction}

People have used the cycles in the sky to keep time since prehistoric ages. Indeed, the relationship between astronomy and timekeeping was instrumental in the development of early mathematics and geometry.

In this lab, you'll explore the choices that different cultures around the world have made in connecting the cycles in the sky to timekeeping. First we will understand the Gregorian calendar, the one we use; then, we will extend that understanding to several other calendars, and finally design one of our own.

\subsection{Cycles in the Sky}

While practices vary throughout cultures in the world, people's approach to timekeeping has centered around the major cycles in the sky.

\begin{enumerate}
	\item {\bf The solar day}: This is the cycle from daytime to nighttime and back; it is 24 hours.
	\item {\bf The sidereal day:} This is the amount of time it takes the Earth to rotate once, and thus for the stars to rotate once in the sky. It is 23.934 hours (23h 56m).
	\item {\bf The synodic month:} This is the time it takes the Moon to cycle through its phases; it is 29.53 solar days.
	\item {\bf The seasonal year:} This is the time from winter solstice to winter solstice. It is 365.2422 solar days.
	\item {\bf The sidereal year:} This is the time it takes the Earth to orbit the Sun once, and thus for the Sun to cycle through the Zodiac. It is 365.2564 solar days.
\end{enumerate}

A calendar, in its most basic form, tells you where you are in the astronomical cycles. For instance, our 24-hour cycle is based on the solar day -- knowing that it is 12:30 PM tells you that it is slightly after noon. Likewise, the Jewish calendar reckons dates based on the phase of the Moon, so if you know that it is the first day of a month, you also know that it is a new moon. 

\section{The Gregorian calendar}

The familiar calendar we use is the Gregorian calendar; we should be familiar with its days, months, and years.

Suppose I tell you that it is 6:29 PM on September 21.  There are three pieces of information here: 

\begin{itemize}
	\item It is 6:29 PM
	\item It is the 21st day of the month
	\item It is September
\end{itemize}

{\bf Question 1.}  Which of the five main cycles in the sky does this connect to? (What does this time/date tell you about daytime vs. nighttime, the position of the stars in their motion around the celestial poles, the phase of the moon, the seasons, and the position of the Sun relative to the Zodiac?) 

(Hint: It is connected to only {\it two} of the five cycles in the sky. Which ones are they, and how are they reflected in the date?)

\vspace{2in} \underline{\hspace{6in}}

{\bf Question 2.} Let's ask the previous question another way. Which of the five cycles in the sky do each of the following describe? 

\begin{enumerate}
	\item The Gregorian day
	\vspace{0.5in}
	\item The Gregorian month
		\vspace{0.5in}
	\item The Gregorian year
\end{enumerate}

\newpage

\subsection{Intercalation (Leap-Things)}

Since the time of year is supposed to tell us what season it is, it would be a problem if this were no longer true.



\end{document}
