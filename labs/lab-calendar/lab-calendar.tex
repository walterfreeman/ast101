\documentclass[11pt]{article}
\usepackage{tocloft}
\usepackage{graphicx}
\usepackage{calc}
\usepackage{amssymb}
\usepackage{color}
\usepackage{array}
\usepackage[sc]{mathpazo}
\usepackage{url}
\usepackage[final]{pdfpages}
\usepackage{amsmath}

%\linespread{1.05}
\oddsidemargin=0pt
\evensidemargin=0pt
\textwidth=6.5in
\topmargin=0pt
\headheight=0pt
\headsep=0pt
\textheight=9in
% EXPERIMENTAL
%\parindent=0pt
%\parskip=3pt
\setlength{\parindent}{0cm}
\newcommand\secfont{\fontfamily{cmss}\selectfont}%\textwidth 5.5truein
\newcommand\pifheading[1]{{\secfont\textbf{#1}:}}
%\oddsidemargin -0.40truein
%\textheight 8.0truein
%\topmargin -0.25truein
\def\lo{
\mathrel{\raise.3ex\hbox{$<$}\mkern-14mu\lower0.6ex\hbox{$\sim$}}
}
\def\hi{
\mathrel{\raise.3ex\hbox{$>$}\mkern-14mu\lower0.6ex\hbox{$\sim$}}
}

\textwidth = 6.6 in
\textheight = 9.1 in
\oddsidemargin = -0.05 in
\evensidemargin = +0.05 in
\topmargin = -.1 in
\headheight = 0.0 in
\headsep = 0.0 in
\parskip = 0.06in
\newcommand\registered{{\ooalign{\hfil\raise .00ex\hbox{\scriptsize R}\hfil\crcr\mathhexbox20D}}}

%% Define a new 'leo' style for the package that will use a smaller font.
\makeatletter
\def\url@leostyle{%
  \@ifundefined{selectfont}{\def\UrlFont{\sf}}{\def\UrlFont{\small\ttfamily}}}
\makeatother
%% Now actually use the newly defined style.
\urlstyle{leostyle}

%\pagestyle{empty}
%\includeonly{previous,proposal_references}
%\includeonly{proposal_references}
%\includeonly{previous}

% TOC

\begin{document}
%%%%%%%%%%%%%%%%%%%%%%%%%%%%%%%%%%%%%%%%%%%%%%%%%%%%%%%%%%%%%%%%%%%%%
\begin{center}
\textbf{\Large
AST101: Our Corner of the Universe \\
\vspace*{0.1cm}
Lab 5: Cycles in the Sky as Clocks
}
\end{center}

\vspace*{0.5cm}

{\Large Name:}\vspace*{0.5cm}\\\hrule
%{\Large Student number (SUID):}\vspace*{0.5cm}\\\hrule
{\Large Lab section:}\vspace*{0.5cm}\\\hrule
{\Large Group Members:}\vspace*{0.5cm}\\\hrule
\vspace*{0.5cm}
\pagenumbering{gobble}
%%%%%%%%%%%%%%%%%%%%%%%%%%%%%%%%%%%%%%%%%%%%%%%%%%%%%%%%%%%%%%%%%%%%%
\section{Introduction}

People have used the cycles in the sky to keep time since prehistoric ages. Indeed, the relationship between astronomy and timekeeping was instrumental in the development of early mathematics and geometry.

In this lab, you'll explore the choices that different cultures around the world have made in connecting the cycles in the sky to timekeeping. First we will understand the Gregorian calendar, the one we use; then, we will extend that understanding to several other calendars, and finally design one of our own.

{\bf A note:} This lab has some arithmetic in it, but no difficult mathematics -- you will need to add, subtract, and multiply, though. So you will need a calculator, smartphone, or a computer's calculator program.

\subsection{Cycles in the Sky}

While practices vary throughout cultures in the world, people's approach to timekeeping has centered around the major cycles in the sky. The most obvious ones are:

\begin{enumerate}
	\item {\bf The solar day}: This is the cycle from daytime to nighttime and back; it is 24 hours.
%	\item {\bf The sidereal (sid-eh-reh-al) day:} This is the amount of time it takes the Earth to rotate once, and thus for the stars to rotate once in the sky. It is 23.934 hours (23h 56m).
	\item {\bf The synodic month:} This is the time it takes the Moon to cycle through its phases; it is 29.53 solar days.
	\item {\bf The seasonal year:} This is the time from winter solstice to winter solstice. It is 365.2422 solar days.
%	\item {\bf The sidereal year:} This is the time it takes the Earth to orbit the Sun once, and thus for the Sun to cycle through the Zodiac. It is 365.2564 solar days.
\end{enumerate}

{\it You will find a reference page on the back of this handout with these numbers -- tear it off.}


%The word ``sidereal'' is based on the Latin ``sider'', meaning ``star''. So a sidereal day is a day by the stars; a solar day is a day by the Sun.

A calendar, in its most basic form, tells you where you are in the astronomical cycles. For instance, our 24-hour cycle is based on the solar day -- knowing that it is 12:30 PM tells you that it is slightly after noon. Likewise, the Jewish calendar reckons dates based on the phase of the Moon, so if you know that it is the first day of a month, you also know that it is a new moon. 



\section{The Gregorian calendar}

The familiar calendar we use is the Gregorian calendar; you are likely familiar with its days, months, and years.

Suppose I tell you that it is 6:29 PM on September 30.  There are three components of the date here: 

\begin{itemize}
	\item It is 6:29 PM
	\item It is the 30th day of the month
	\item It is September
\end{itemize}

Which of the three main cycles in the sky does this connect to? (What does this time/date tell you about daytime vs. nighttime, the phase of the moon, and the seasons?) 

(Hint: It is connected to only {\it two} of these three cycles; one of the components of the date doesn't actually carry any information! Which cycles are they, and how are they reflected in the date?)

\vspace{2in} \underline{\hspace{6in}}

Let's ask the previous question another way. Which of the three cycles in the sky do each of the following describe? (One of them doesn't describe any of them!)

\begin{enumerate}
	\item The Gregorian day
	\vspace{0.5in}
	\item The Gregorian month
		\vspace{0.5in}
	\item The Gregorian year
\end{enumerate}

\newpage

\subsection{Intercalation (Leap-Things)}

Our goal here is to understand why we add leap days. 

Suppose that we stopped putting leap days in the Gregorian calendar -- if every year was 365 solar days long. What do you think would happen over the next few hundred years? How long would it take before people noticed that something was wrong?

\vspace{1.5in}

Our pattern of leap-days repeats every 400 years, so let's use that for reference. 
\vspace{1em}

\begin{minipage}{0.45\textwidth}
How many solar days are in 400 seasonal years? (Remember, one seasonal year is 365.2422 solar days.)

\vspace{1in}

\underline{\hspace{3in}}
\end{minipage}
\hspace{0.1\textwidth}
\begin{minipage}{0.45\textwidth}
	How many solar days are in 400 years if the year is exactly 365 days (with no leap years at all)
	
	\vspace{1in}
	
	\underline{\hspace{3in}}
\end{minipage}

\vspace{1in}

How big of a deal do you think this discrepancy is? 

\vspace{2in}

	\underline{\hspace{6in}}
	
	\newpage

In 45 BC, Julius Caesar issued an edict to add an {\it intercalary day} or {\it leap day} every fourth year, so the Julian calendar had three years of 365 days, followed by one year of 366 days.

How many days are in 400 Julian years? {\it (Hint: You just calculated the number of days in 400 365-day years; you can just add the number of leap days in 400 years to this.)}

\vspace{1in}

\underline{\hspace{6in}}

How does this compare to the number of days in 400 seasonal years? How long do you think before people would notice the remaining discrepancy?

\vspace{1in}
\underline{\hspace{6in}}

Around 1600, Pope Gregory introduced a new rule leading to the calendar we use today: years ending in 00 would not be leap years, unless they were divisible by 400. So 1600 would be a leap year, but 1700, 1800, and 1900 would not be; 2000 would be a leap year.

How close does this scheme come to aligning the Gregorian calendar with the seasonal year? (How close are 400 Gregorian years to 400 seasonal years? Compare the number of days Gregory added to the discrepancy that you determined in the previous question.)


\vspace{1in}
\underline{\hspace{6in}}
\newpage
\section{The Islamic (Hijra) Calendar}

You'll notice that we totally ignored the 29.53-day cycle of the moon phases in the Gregorian calendar, which is based only on the seasonal year. The Gregorian months are {\it close} to ``moonths'', but they are longer by a few days so that the cycle of the moon phases doesn't align with the Gregorian months.

The Islamic calendar, on the other hand, is based extremely heavily on the cycles of the Moon. It is based on two principles:

\begin{enumerate}
	\item Every month should begin on the new moon (it will be either 29 or 30 days, since the lunar cycle is 29.53 days)
	\item There should be exactly 12 months in a year (there is a Quranic verse prohibiting adding extra months)
\end{enumerate}

How many days is a year in the Islamic calendar? How does this compare to the number of days in a seasonal year?

\vspace{1.2in}
\underline{\hspace{6in}}

An important day in the Islamic calendar is Eid al-Fitr, the end of the month of Ramadan. It is the first day of the month of Shawwal in the Islamic calendar, but it falls on different dates in the Gregorian calendar:



\begin{minipage}{0.5\textwidth}
	\begin{center}
		\begin{tabular}{|c|c|}
			\hline
			Year & Date of Eid al-Fitr \\ \hline
			2001 & 16 December         \\ \hline
			2002 & 5 December          \\ \hline
			2003 & 25 November         \\ \hline
			2004 & 14 November         \\ \hline
			2005 & 3 November          \\ \hline
			2006 & 23 October          \\ \hline
			2007 & 13 October          \\ \hline
			2008 & 1 October           \\ \hline
			2009 & 20 September        \\ \hline
			2010 & 10 September        \\ \hline
			2011 & 30 August           \\ \hline
		
		\end{tabular}
	\end{center}
\end{minipage}
\begin{minipage}{0.5\textwidth}
	\begin{center}
	\begin{tabular}{|c|c|}
		\hline
		Year & Date of Eid al-Fitr \\ \hline
	
		2012 & 19 August           \\ \hline
		2013 & 8 August            \\ \hline
		2014 & 28 July             \\ \hline
		2015 & 17 July             \\ \hline
		2016 & 6 July              \\ \hline
		2017 & 25 June             \\ \hline
		2018 & 15 June             \\ \hline
		2019 & 4 June              \\ \hline
		2020 & 24 May              \\ \hline
		2021 & 13 May              \\ \hline
		2022 & 2 May               \\ \hline
	\end{tabular}
\end{center}
\end{minipage}

\normalsize

\newpage

What do you think is going on here? Based on the definition of the Islamic calendar (12 lunar months per year), why does Eid al-Fitr ``drift'' backwards relative to the Gregorian calendar?

\vspace{1.6in}
\underline{\hspace{6in}}


Muslims observe the month of Ramadan by not eating or drinking from sunrise to sunset. This tradition, and the Islamic calendar, were devised in the tropics. Based on what you have learned about the seasons and the way they vary across Earth, would the experience of observing the Ramadan fast differ from place to place and in different years?


\vspace{2in}
\underline{\hspace{6in}}

\section{The Chinese and Jewish calendars}

As we have seen, the Gregorian calendar has a year synchronized (within a day) to the cycle of seasons, but ignores the Moon; the Islamic calendar synchronizes its twelve months to the Moon, but ignores the seasons. 

This results in the Islamic year ``slipping'' 11 days relative to the cycle of the seasons each year.

But what if a culture wanted to observe both the seasons and the Moon? For instance, the Jewish holiday of Rosh Hashanah marks the new year; it is both always on a new moon and always during Northern Hemisphere autumn. The Chinese calendar works in a similar way, with the Lunar New Year happening on a new moon during late Northern Hemisphere winter. The Chinese and Jewish calendars are thus {\it lunisolar} -- they care about both the Sun and the Moon.

\newpage
Unlike in Islam, Jewish tradition contains no prohibition on adding or removing months from a year. How might Jewish timekeepers ensure that Rosh Hashanah always happens in the early fall (in the Northern Hemisphere), or Chinese timekeepers keep the New Year in the late winter?

\vspace{2in}
\underline{\hspace{6in}}

As you saw in the Islamic calendar, 12 lunar months is 354.36 days - about 11 days short of the seasonal year. How many days is a ``leap year'' in the Jewish or Chinese system, and how does it compare to the seasonal year?

\vspace{1in}
\underline{\hspace{6in}}


Jewish and Chinese timekeepers' approach to the months repeats every 19 years, so we'll use this as a base for our calculation. 

\bigskip

\begin{minipage}{0.45\textwidth}
	How many solar days are in 19 seasonal years? (Remember the seasonal year is 365.2422 solar days.)
	
	\vspace{1in}
	
	\underline{\hspace{3in}}
\end{minipage}
\hspace{0.1\textwidth}
\begin{minipage}{0.45\textwidth}
	How many lunar months is this? (Remember the lunar month is 29.53 days.) Does it come out close to even?
	
	\vspace{1in}
	
	\underline{\hspace{3in}}
\end{minipage}

In a 19-year-cycle, how many intercalary (leap) months must be added in total, so that 19 Jewish/Chinese calendar years equal 19 seasonal years? This 19-year cycle, discovered independently by various cultures that tracked both the lunar and seasonal cycles, is called the Metonic cycle.

\vspace{1in}
\underline{\hspace{6in}}

\begin{minipage}{0.6\textwidth}
	Here is a table of the Gregorian dates of Rosh Hashanah in recent years, spanning one complete 19-year Metonic cycle.
	
	\bigskip
	
	What patterns do you see? How are they explained by the lunisolar system that mixes 12-lunar-month and 13-lunar-month years?
\vspace{2in}

\end{minipage}
\hspace{0.1\textwidth}
\begin{minipage}{0.3\textwidth}
	\begin{center}
		\begin{tabular}{|c|c|}
			\hline
			
			Year & Date of Rosh Hashanah \\ \hline
			2002 & 7 September              \\ \hline
			2003 & 27 September              \\ \hline
			2004 & 16 September              \\ \hline
			2005 & 4 October             \\ \hline
			2006 & 23 October             \\ \hline
			2007 & 13 September              \\ \hline
			2008 & 30 September              \\ \hline
			2009 & 19 September              \\ \hline
			2010 & 9 September          \\ \hline
			2011 & 29 September            \\ \hline
			2012 & 17 September             \\ \hline
			2013 & 5 September             \\ \hline
			2014 & 25 September              \\ \hline
			2015 & 14 September             \\ \hline
			2016 & 3 October             \\ \hline
			2017 & 21 September              \\ \hline
			2018 & 10 September              \\ \hline
			2019 & 30 September              \\ \hline
			2020 & 19 September               \\ \hline
			2021 & 7 September               \\ \hline
		\end{tabular}
	\end{center}
\end{minipage}


\vspace{1in}
\underline{\hspace{6in}}

\section{Time by the Stars}

So far, we've seen three approaches:

\begin{enumerate}
	\item The Gregorian calendar ignores the cycle of the Moon entirely; the year is very close to the seasonal year, with a pattern of 365 and 366 day years so that the calendar year matches the seasonal year on average
	\item The Islamic calendar ignores the seasons entirely; the year is 12 lunar months (29 or 30 days)
	\item The Jewish and Chinese calendars consider both; the year is 12 or 13 lunar months, so that the calendar year matches the seasonal year on average
\end{enumerate}

These are focused on the Sun and Moon. But there are two more cycles in the sky we have ignored that involve the stars themselves:
\newpage
\begin{enumerate}
	\item The {\it sidereal year}, the time it takes for the Sun to pass through all of the constellations in the Zodiac. This takes 365.2564 days.
	\item The {\it sidereal day}, the time it takes Earth to rotate once on its axis (and thus for the stars to rotate once around the celestial sphere and come back to where they started). This takes 23~hours 56 minutes.
\end{enumerate}

\subsection{The Astrological/Zodiac Calendar and the Sidereal Year}

Ancient astronomers divided the Sun's path against the stars into twelve equal pieces, and named each after a mythological character -- these are the twelve constellations in the Zodiac.

How long does it take for the Sun to pass through each constellation? How does this compare to other subdivisions of the year that you know about? (Why do you think astronomers divided the Zodiac into twelve pieces, rather than some other number?)

\vspace{1.8in}
\underline{\hspace{6in}}

We now have {\it two} ways to define a year: 

\begin{enumerate}
	\item The cycle of the seasons (seasonal year): 365.2422 days
	\item The cycle of the Sun in the zodiac (sidereal year): 365.2564 days
\end{enumerate}

These are very slightly different because the direction that Earth's axis is tilted changes slowly over time. 

Does it make sense to define a new year by the seasons (``the new year starts in the winter'') or by the stars (``the new year starts when the Sun is in Aries'')? Do {\it both} choices make sense? Is there a reason that some ancient cultures might have favored one or the other?


\vspace{1.8in}
\underline{\hspace{6in}}

Ancient astronomers observed that the motion of the Sun through the Zodiac could serve as a way to mark the progress of time through the sidereal year, and divided the stars in the Zodiac into twelve equal sections to form the astrological calendar. 

What effects does the slight difference here have? To see, let's indulge in a bit of whimsy.

Suppose that two ancient cultures on different sides of a river made different choices in their timekeeping. The Austrads, living south of the sea, chose to track the sidereal year; the Boreads, living to the north, chose to track the seasonal year. 

As friendly neighbors, they decided to meet once a year at on the banks of the river and celebrate the precise moment of the New Year together. Two young astronomers from the two cultures met to sort out the details. 

\vspace{1em}

\hspace{0.05\textwidth}
\begin{minipage}{0.9\textwidth}
{\bf Boread astronomer:} ``Our people observe the seasons. So we want to have our New Year on the winter solstice.''

\medskip

{\bf Austrad astronomer:} ``That works for us. Looking at the stars, I see that's when the Sun passes from Scorpius to Sagittarius.''

\medskip

{\bf Boread astronomer:} ``So we agree, then. We will come back here in one year on the next winter solstice, and you can bring your people in one year when the Sun enters Sagittarius.''
\end{minipage}

One year later, the two peoples meet on the banks of the river. Would they notice anything unusual because of their different ways of defining a year?


%\underline{\hspace{6in}}
\newpage
Sadly, the Austrads and Boreads aren't able to observe their tradition again for a long time. However, after a hundred years, they decide to do this again, and meet at the river. The Austrads come back after exactly a hundred sidereal years, when the Sun again enters Sagittarius; the Boreads come back after exactly a hundred seasonal years, on the winter solstice.

Will they get to see their neighbors on the New Year? If not, who will be late, and by how much? {\it (Try calculating the number of days in a hundred years for each culture!)}

\vspace{3in}
\underline{\hspace{6in}}

\subsection{Sidereal vs. Traditional Astrology}

In Lab 1, you observed that your horoscope sign is ``wrong''. This story explains why that is! 

Long ago, people noted the dates on the seasonal calendar (defined relative to the solstices and equinoxes) that the Sun was in front of each constellation in the Zodiac. These are the dates that are commonly used in horoscopes.

However, because of the slight mismatch between the seasonal year and the sidereal year, the astrological calendar (using the stars) and the Gregorian calendar (using the seasons) have gotten out of sync: the seasonal-calendar dates written down long ago that tell when the Sun is in each constellation are wrong! 

The chart on the next page shows the current situation.

% Please add the following required packages to your document preamble:
% \usepackage[table,xcdraw]{xcolor}
% If you use beamer only pass "xcolor=table" option, i.e. \documentclass[xcolor=table]{beamer}
% Please add the following required packages to your document preamble:
% \usepackage[table,xcdraw]{xcolor}
% If you use beamer only pass "xcolor=table" option, i.e. \documentclass[xcolor=table]{beamer}
\begin{center}
	\begin{tabular}{ c c c }
		\hline
		             & Astrological Dates &                          \\ 
		Constellation \hspace{0.5in} & (Date Behind Sun Long Ago) & Date Behind Sun At Present                         \\ \hline
		& & \\ 
		Aries     \hspace{0.5in}  & March 21 - April 19       & April 15 - May 15         \\ 
		Taurus    \hspace{0.5in}  & April 20 - May 20         & May 16 - June 15          \\ 
		Gemini    \hspace{0.5in}  & May 21 - June 20          & June 16 - July 10         \\ 
		Cancer    \hspace{0.5in}  & June 21 - July 22         & July 11 - August 16       \\ 
		Leo        \hspace{0.5in} & July 23 - August 22       & August 17 - September 16  \\ 
		Virgo     \hspace{0.5in}  & August 23 - September 22  & September 17 - October 17 \\ 
		Libra      \hspace{0.5in} & September 23 - October 22 & October 18 - November 16  \\ 
		Scorpio    \hspace{0.5in} & October 23 - November 21  & November 17 - December 16 \\ 
		Sagittarius\hspace{0.5in} & November 22 - December 21 & December 17 - January 15  \\ 
		Capricorn \hspace{0.5in}  & December 22 - January 19  & January 16 - February 14  \\ 
		Aquarius   \hspace{0.5in} & January 20 - February 19  & February 15 - March 15    \\
		Pisces     \hspace{0.5in} & February 19 - March 20    & March 16 - April 14       \\  
		&                         &                         \\ \hline
	\end{tabular}
\end{center}

So, for instance, the Sun enters Aries on April 15 currently. But, long ago when the dates used in horoscopes were written down, the Sun entered Aries on March 21 (25 days earlier). 

How long ago was that?

\vspace{2in}
\underline{\hspace{6in}}


{\it A note: Some ancient astronomers knew about this effect -- to notice it and calculate its size, a culture would need to make detailed observations of the solstices and the stars for a long time. Indian astronomers noticed this thousands of years ago -- I have seen suggestions that they had documented this as early as 2500 BCE. The ancient astrological calendar was not ``wrong'' -- it matched the stars exactly when it was written down! If anyone is ``wrong'', it is {\it modern} Western astrology, for not correcting for the difference between the sidereal and seasonal years! Hindu astrology {\it} does correct for this (after all, the Indians have known about it for a very long time).}

\subsection{The Sidereal Day}

I don't know any cultures that use the sidereal {\it day}, rather than the solar day, on their calendars. But perhaps we can think of one! What sort of fictional culture would observe the sidereal day, rather than the solar day, for at least part of their year?


\newpage

\Large

\begin{center}
	
	Reference
	\vspace{3in}	

	\begin{tabular}{ccc}
		\bf Cycle & \bf Definition & \bf Time \\ \hline
		 & & \\ 
		Solar day & From noon to noon & 24 hours            \\
		Synodic month                 & Cycle of moon phases                  & 29.53 solar days    \\
		Seasonal year                 & Cycle of seasons                      & 365.2422 solar days \\
		Sidereal year                 & Cycle of Sun through Zodiac           & 365.2564 solar days \\
		Sidereal day                  & Apparent motion of stars              & 23 hours 56 minutes
	\end{tabular}

\end{center}

\end{document}


