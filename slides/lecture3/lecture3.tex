\documentclass[10pt]{beamer}
\usepackage{amsmath}
\usepackage{mathtools}
\usepackage{multimedia}
\usepackage{hyperref}


\usefonttheme{professionalfonts} % using non standard fonts for beamer
\usefonttheme{serif} % default family is serif
%\documentclass[12pt]{beamerthemeSam.sty}
\usepackage{epsf}
%\usepackage{pstricks}
%\usepackage[orientation=portrait,size=A4]{beamerposter}
\geometry{paperwidth=160mm,paperheight=120mm}
%DT favorite definitions
\def\LL{\left\langle}	% left angle bracket
\def\RR{\right\rangle}	% right angle bracket
\def\LP{\left(}		% left parenthesis
\def\RP{\right)}	% right parenthesis
\def\LB{\left\{}	% left curly bracket
\def\RB{\right\}}	% right curly bracket
\def\PAR#1#2{ {{\partial #1}\over{\partial #2}} }
\def\PARTWO#1#2{ {{\partial^2 #1}\over{\partial #2}^2} }
\def\PARTWOMIX#1#2#3{ {{\partial^2 #1}\over{\partial #2 \partial #3}} }

\def\rightpartial{{\overrightarrow\partial}}
\def\leftpartial{{\overleftarrow\partial}}
\def\diffpartial{\buildrel\leftrightarrow\over\partial}

\def\BC{\begin{center}}
\def\BS{\bigskip}
\def\EC{\end{center}}
\def\BN{\begin{enumerate}}
\def\EN{\end{enumerate}}
\def\BI{\begin{itemize}}
\def\EI{\end{itemize}}
\def\BE{\begin{displaymath}}
\def\EE{\end{displaymath}}
\def\BEA{\begin{eqnarray*}}
\def\EEA{\end{eqnarray*}}
\def\BNEA{\begin{eqnarray}}
\def\ENEA{\end{eqnarray}}
\def\EL{\nonumber\\}

\newcommand{\etal}{{\it et al.}}
\newcommand{\gbeta}{6/g^2}
\newcommand{\la}[1]{\label{#1}}
\newcommand{\ie}{{\em i.e.\ }}
\newcommand{\eg}{{\em e.\,g.\ }}
\newcommand{\cf}{cf.\ }
\newcommand{\etc}{etc.\ }
\newcommand{\atantwo}{{\rm atan2}}
\newcommand{\Tr}{{\rm Tr}}
\newcommand{\dt}{\Delta t}
\newcommand{\op}{{\cal O}}
\newcommand{\msbar}{{\overline{\rm MS}}}
\def\chpt{\raise0.4ex\hbox{$\chi$}PT}
\def\schpt{S\raise0.4ex\hbox{$\chi$}PT}
\def\MeV{{\rm Me\!V}}
\def\GeV{{\rm Ge\!V}}

%AB: my color definitions
%\definecolor{mygarnet}{rgb}{0.445,0.184,0.215}
%\definecolor{mygold}{rgb}{0.848,0.848,0.098}
%\definecolor{myg2g}{rgb}{0.647,0.316,0.157}
\definecolor{A}{rgb}{1.0,0.3,0.3}
\definecolor{B}{rgb}{0.0,1.0,0.0}
\definecolor{C}{rgb}{1.0,1.0,0.0}
\definecolor{D}{rgb}{0.5,0.5,1.0}
\definecolor{E}{rgb}{0.7,0.7,0.7}
\definecolor{abtitlecolor}{rgb}{1.0,1.0,1.0}
\definecolor{absecondarycolor}{rgb}{0.0,0.416,0.804}
\definecolor{abprimarycolor}{rgb}{1.0,0.686,0.0}
\definecolor{Red}           {rgb}{1,0.4,0.4}
\definecolor{Yellow}           {rgb}{1,1,0.0}
\definecolor{Grey}          {cmyk}{.7,.7,.7,0}
\definecolor{Blue}          {cmyk}{1,1,0,0}
\definecolor{Green}         {cmyk}{1,0,1,0}
\definecolor{Brown}         {cmyk}{0,0.81,1,0.60}
\definecolor{Silver}        {rgb}{0.95,0.9,1.0}
\definecolor{Sky}           {rgb}{0.07,0.0,0.2}
\definecolor{Darkbrown}     {rgb}{0.4,0.3,0.2}
\definecolor{40Gray}        {rgb}{0.4,0.4,0.5}
\usetheme{Madrid}


\setbeamercolor{normal text}{fg=Silver,bg=Sky}

%AB: redefinition of beamer colors
%\setbeamercolor{palette tertiary}{fg=white,bg=mygarnet}
%\setbeamercolor{palette secondary}{fg=white,bg=myg2g}
%\setbeamercolor{palette primary}{fg=black,bg=mygold}
\setbeamercolor{title}{fg=abtitlecolor}
\setbeamercolor{frametitle}{fg=abtitlecolor}
\setbeamercolor{palette tertiary}{fg=white,bg=Darkbrown}
\setbeamercolor{palette secondary}{fg=white,bg=absecondarycolor}
\setbeamercolor{palette primary}{fg=white,bg=40Gray}
\setbeamercolor{structure}{fg=abtitlecolor}

\setbeamerfont{section in toc}{series=\bfseries}

%AB: remove navigation icons
\beamertemplatenavigationsymbolsempty
\title[The daily motion of the sky]{
  \textbf {The daily motion of the sky}
}

\author [Astronomy 101]{Astronomy 101\\Syracuse University, Fall 2018\\Walter Freeman}

\date{\today}

\begin{document}



\frame{\titlepage}


\frame{\frametitle{\textbf{Some announcements}}
\Large
\BI
\item{While I was out of touch over the weekend, I've gotten a lot of mail.}
\item{I'll have all that answered by the end of the day today.}
\item{Labs start this week!}
\EI
}


\frame{\frametitle{\textbf{Opportunities for help}}
\Large

\BI
\item AST101 help sessions: Wednesday 3-5PM, or Friday 9:30-11:30AM, in my office or the Physics Clinic if we run out of room
\pause
\item Other times in the Clinic during business hours (8AM-9PM)
\pause
\item Appointments with me by request
\pause 
\item By email: most of you have cellphone cameras...
\EI
}


\frame{\frametitle{\textbf{The lecture tutorials}}
\Large

We won't be ``going over'' them in class -- this deprives you of an opportunity.

\bigskip
\bigskip
\bigskip
\pause

Remember, the goal of the tutorials isn't for you to learn {\it those} things -- it's to give you
practice in reasoning so you can figure out {\it other} things.

\bigskip
\bigskip
\bigskip

\pause

So I am eager to discuss them with you in the help sessions, and your TA's can help you as well. 

\pause

\BS
If you're not sure about the reasoning involved, {\it ask us in class}. We won't tell you the answers
but we'll make sure you know how to get there on your own! (That's the point...)
}


\frame{\frametitle{\textbf{This week we will...}}
\Large
Today: consequences of the Earth's {\bf rotation}:
\BI
\item{Review the celestial-sphere model from last time}
\item{Look in more detail at its consequences}
\item{Complete the first {\it Lecture Tutorial} that we started last time}
\EI
\bigskip
\bigskip
Thursday: consequences of the Earth's {\bf revolution}:
\BI
\item{What about the Sun?}
\item{What causes the seasons?}
\item{What does the Sun do to the Earth?}
\EI
}

\frame{\frametitle{\textbf{Which are true in Syracuse?}}
\begin{columns}
\begin{column}{0.6\textwidth}
\Large
\BI
\item{I: Some stars are always visible (at night)}
\item{II: Some stars are only visible sometimes; they rise and set during the night}
\item{III: Some stars are never visible}
\EI
\end{column}
\begin{column}{0.4\textwidth}
\includegraphics[width=\textwidth]{sphere-3.png}
\end{column}
\end{columns}
\bigskip
\bigskip
\Large
\color{A}A: I only \\
\color{B}B: II only \\
\color{C}C: III only \\
\color{D}D: I and II \\
\color{E}E: I, II, and III 
}


\frame{\frametitle{\textbf{Summary}}
\large
\BI
\item{We can treat the stars as all rotating together, on an invisible sphere far away}
\item{The axis of rotation is the same as the Earth's, and it rotates once per day}
\item{Only half of the sphere is visible, because the Earth is in the way}
\EI
\BC\includegraphics[width=0.5\textwidth]{sphere-3.png}\EC 
}

\frame{
\Large


This picture was taken in Australia. Which way is the photographer looking?

\bigskip
\bigskip
\begin{columns}
\begin{column}{0.3\textwidth}
\Huge
\color{A}A: North \\
\color{B}B: South \\
\color{C}C: East \\
\color{D}D: West \\
\end{column}
\begin{column}{0.7\textwidth}
\BC
\includegraphics[width=\textwidth]{outback.jpg} \EC
\end{column}
\end{columns}
}

\frame{
\Large

In Syracuse, you see this star at (A). Where will it be six hours later?

\BC
\includegraphics[width=0.8\textwidth]{star-question.png}
\EC
}

\frame{

\BC
\Large
Which star is visible longer?

\bigskip
\bigskip

\includegraphics[width=0.8\textwidth]{visible-longer-1.png}
\EC

\pause\bigskip

We call a star that is always above the horizon {\color{Red}circumpolar}.
}	

\frame{
\BC
\Large

Where in the sky should I look to find circumpolar stars?

\EC

\BS\BS
\color{A}A: High in the southern sky \\
\color{B}B: Low in the eastern sky \\
\color{C}C: High in the northern sky \\
\color{D}D: Low-ish in the northern sky \\
\color{E}E: Low in the northern sky \\
}

\frame{
\Large
\BC
What about now? Which star is visible longer?

\bigskip
\bigskip

\includegraphics[width=0.8\textwidth]{visible-longer-2.png}
\EC
}	

\frame{\frametitle{\textbf{Lecture tutorials}}
\Huge
\BC
Complete pages 1-6.

\Large
\bigskip
\bigskip
We will talk about something else after this.
\EC
}

\frame{\frametitle{\textbf{What about the Sun?}}
Over the course of one day, the Sun doesn't move very much.

\bigskip
\bigskip

This means the celestial sphere model can show us how the Sun moves {\color{Red} each day}.

\bigskip
\bigskip

It rises in the East and sets in the West, just like all the other stars.

}

\frame{
\BC
\Large
How long is this star visible in the sky each day?
\bigskip
\bigskip

\EC
\begin{columns}
\column{0.4\textwidth}
\Large
\color{A}A: All the time \\
\color{B}B: More than 12 hours \\
\color{C}C: Less than 12 hours \\
\color{D}D: It's never visible  
\column{0.6\textwidth}
\BC
\includegraphics[width=0.95\textwidth]{how-long-1.png}
\EC
\end{columns}

}

\frame{
\BC
\Large
How long is this star visible in the sky each day?
\bigskip
\bigskip

\EC
\begin{columns}
\column{0.4\textwidth}
\Large
\color{A}A: All the time \\
\color{B}B: More than 12 hours \\
\color{C}C: Less than 12 hours \\
\color{D}D: It's never visible  
\column{0.6\textwidth}
\BC
\includegraphics[width=0.95\textwidth]{how-long-2.png}
\EC
\end{columns}

}

\frame{
\BC
\Large
How long is this star visible in the sky each day?
\bigskip
\bigskip

\EC
\begin{columns}
\column{0.4\textwidth}
\Large
\color{A}A: All the time \\
\color{B}B: More than 12 hours \\
\color{C}C: Less than 12 hours \\
\color{D}D: It's never visible  
\column{0.6\textwidth}
\BC
\includegraphics[width=0.95\textwidth]{how-long-3.png}
\EC
\end{columns}

}

\frame{
\BC
\Large
How long is this star visible in the sky each day?
\bigskip
\bigskip

\EC
\begin{columns}
\column{0.4\textwidth}
\Large
\color{A}A: All the time \\
\color{B}B: More than 12 hours \\
\color{C}C: Less than 12 hours \\
\color{D}D: It's never visible  
\column{0.6\textwidth}
\BC
\includegraphics[width=0.95\textwidth]{how-long-4.png}
\EC
\end{columns}

}


\frame{\frametitle{\textbf{What does one star matter?}}
\Large
The visibility of one star in our sky isn't that big of a deal...

\pause
\bigskip
\bigskip

... unless that star is the Sun! We'll talk about this Thursday.
}



\end{document}
