
\documentclass[12pt]{beamer}
\usepackage{amsmath}
\usepackage{mathtools}
\usepackage{multimedia}
\usepackage{hyperref}


\usefonttheme{professionalfonts} % using non standard fonts for beamer
\usefonttheme{serif} % default family is serif
%\documentclass[12pt]{beamerthemeSam.sty}
\usepackage{epsf}
%\usepackage{pstricks}
%\usepackage[orientation=portrait,size=A4]{beamerposter}
\geometry{paperwidth=160mm,paperheight=120mm}
%DT favorite definitions
\def\LL{\left\langle}	% left angle bracket
\def\RR{\right\rangle}	% right angle bracket
\def\LP{\left(}		% left parenthesis
\def\RP{\right)}	% right parenthesis
\def\LB{\left\{}	% left curly bracket
\def\RB{\right\}}	% right curly bracket
\def\PAR#1#2{ {{\partial #1}\over{\partial #2}} }
\def\PARTWO#1#2{ {{\partial^2 #1}\over{\partial #2}^2} }
\def\PARTWOMIX#1#2#3{ {{\partial^2 #1}\over{\partial #2 \partial #3}} }

\def\rightpartial{{\overrightarrow\partial}}
\def\leftpartial{{\overleftarrow\partial}}
\def\diffpartial{\buildrel\leftrightarrow\over\partial}

\def\BCC{\begin{columns}}
\def\ECC{\end{columns}}
\def\HC{\column{0.5\textwidth}}
\def\BC{\begin{center}}
\def\EC{\end{center}}
\def\BN{\begin{enumerate}}
\def\EN{\end{enumerate}}
\def\BI{\begin{itemize}}
\def\EI{\end{itemize}}
\def\BE{\begin{displaymath}}
\def\EE{\end{displaymath}}
\def\BEA{\begin{eqnarray*}}
\def\EEA{\end{eqnarray*}}
\def\BNEA{\begin{eqnarray}}
\def\ENEA{\end{eqnarray}}
\def\EL{\nonumber\\}

\newcommand{\etal}{{\it et al.}}
\newcommand{\gbeta}{6/g^2}
\newcommand{\la}[1]{\label{#1}}
\newcommand{\ie}{{\em i.e.\ }}
\newcommand{\eg}{{\em e.\,g.\ }}
\newcommand{\cf}{cf.\ }
\newcommand{\BS}{\bigskip}
\newcommand{\etc}{etc.\ }
\newcommand{\atantwo}{{\rm atan2}}
\newcommand{\Tr}{{\rm Tr}}
\newcommand{\dt}{\Delta t}
\newcommand{\op}{{\cal O}}
\newcommand{\msbar}{{\overline{\rm MS}}}
\def\chpt{\raise0.4ex\hbox{$\chi$}PT}
\def\schpt{S\raise0.4ex\hbox{$\chi$}PT}
\def\MeV{{\rm Me\!V}}
\def\GeV{{\rm Ge\!V}}

%AB: my color definitions
%\definecolor{mygarnet}{rgb}{0.445,0.184,0.215}
%\definecolor{mygold}{rgb}{0.848,0.848,0.098}
%\definecolor{myg2g}{rgb}{0.647,0.316,0.157}
\definecolor{A}{rgb}{1.0,0.3,0.3}
\definecolor{B}{rgb}{0.0,1.0,0.0}
\definecolor{C}{rgb}{1.0,1.0,0.0}
\definecolor{D}{rgb}{0.5,0.5,1.0}
\definecolor{E}{rgb}{0.7,0.7,0.7}
\definecolor{abtitlecolor}{rgb}{1.0,1.0,1.0}
\definecolor{absecondarycolor}{rgb}{0.0,0.416,0.804}
\definecolor{abprimarycolor}{rgb}{1.0,0.686,0.0}
\definecolor{Red}           {rgb}{1,0.4,0.4}
\definecolor{Yellow}           {rgb}{1,1,0.0}
\definecolor{Grey}          {cmyk}{.7,.7,.7,0}
\definecolor{Blue}          {cmyk}{1,1,0,0}
\definecolor{Green}         {cmyk}{1,0,1,0}
\definecolor{Brown}         {cmyk}{0,0.81,1,0.60}
\definecolor{Silver}        {rgb}{0.95,0.9,1.0}
\definecolor{Sky}           {rgb}{0.07,0.0,0.2}
\definecolor{Darkbrown}     {rgb}{0.4,0.3,0.2}
\definecolor{Black}         {rgb}{0.0,0.0,0.0}
\definecolor{40Gray}        {rgb}{0.4,0.4,0.5}
\usetheme{Madrid}


\setbeamercolor{normal text}{fg=Silver,bg=Sky}

%AB: redefinition of beamer colors
%\setbeamercolor{palette tertiary}{fg=white,bg=mygarnet}
%\setbeamercolor{palette secondary}{fg=white,bg=myg2g}
%\setbeamercolor{palette primary}{fg=black,bg=mygold}
\setbeamercolor{title}{fg=abtitlecolor}
\setbeamercolor{frametitle}{fg=abtitlecolor}
\setbeamercolor{palette tertiary}{fg=white,bg=Darkbrown}
\setbeamercolor{palette secondary}{fg=white,bg=absecondarycolor}
\setbeamercolor{palette primary}{fg=white,bg=40Gray}
\setbeamercolor{structure}{fg=abtitlecolor}

\setbeamerfont{section in toc}{series=\bfseries}

%AB: remove navigation icons
\beamertemplatenavigationsymbolsempty
\title[The Sun]{
  \textbf {The Sun}}

\author [Astronomy 101]{Astronomy 101\\Syracuse University, Fall 2022\\Walter Freeman}

\date{\today}

\begin{document}



\frame{\titlepage}

\frame{
	
	\large
	
	Dead the new astronomy calls her, ... \\
	Dead, but how her living glory lights the fall, the dune, the grass! \\
	Yet the moonlight is the sunlight, and the sun himself will pass. 
	
	\BS
	
	\begin{flushright}
		\it --Alfred, Lord Tennyson (1886)
	\end{flushright}




%There is a doctrine well known to philosophers that the moon ceases to exist when no one is looking at it. I will not discuss the doctrine since I have not the least idea what is the meaning of the word existence when used in this connection. At any rate the science of astronomy has not been based on this spasmodic kind of moon. In the scientific world (which has to fulfill functions less vague than merely existing) there is a moon which appeared on the scene before the astronomer; it reflects sunlight when no one sees it; it has mass when no one is measuring the mass; it is distant 240,000 miles from the earth when no one is surveying the distance; and it will eclipse the sun in 1999 even if the human race has succeeded in killing itself off before that date.
%— Eddington, The Nature of the Physical World, 226

}

\frame{\frametitle{\textbf{Announcements}}
\Large
\BI
\item {\color{Red}Prelabs for next week are in the Physics Clinic}
\item We are working on grading your Exam 3's; you should get them back next week
\item We are also working on grading your papers and retake/makeup homework quizzes
\BS
\BS
\pause
\item Don't forget your takehome labs
\item A reminder about final projects
\EI

}
%
%\frame{\frametitle{\textbf{Last year's Exam 3}}
%\large
%Last year's exam has some questions on it that you may not know how to do yet, or that may not be part of Unit 3 this year:
%
%
%\BI
%\item Questions 4 and 9 cover neutrino astronomy, which was on Unit 2 this year
%\item Question 29 covers conservation of angular momentum, which will be on Unit 4 this year
%\item Questions 23 and 24 cover what will be next week's lab, which will be on Unit 4 this year
%\pause
%\BS
%\item Question 7 covers today's material on how the Sun works
%\pause
%\BS
%\item Questions 8, 11, 13, 19, 25, 26, and 30 touch on an idea we'll review now
%\EI
%}
%
%\frame{\frametitle{\textbf{Benchmarks: thermal radiation}}
%\large
%You should know, roughly, what sort of light objects at different temperatures emit.
%
%\BS
%
%Keep in mind that an object may {\it mostly} emit light outside our visible range...
%
%\BS
%
%\small
%\begin{tabular}{|l|l|l|l|}
%	\hline
%	& \color{Red}T (Kelvin) & \color{Red}Peak wavelength & \color{Red}Visible light?                          \\ \hline
%	Deep space                 & 3                    & Microwave       & None                                    \\ \hline
%	Freezing ice               & 273                  & Infrared        & None                                    \\ \hline
%	People                     & 300                  & Infrared        & None                                    \\ \hline
%	Boiling water              & 373                  & Infrared        & None                                    \\ \hline
%	Hot stove, candle, etc.      & 1500                 & Near infrared   & A little red               \\ \hline
%	Incandescent light         & 2400-2800            & Near infrared   & Mostly red/orange       \\ \hline
%	The Sun                    & 5700                 & Visible         & Mix of all colors (looks white)  \\ \hline
%	Hot stars & 10000+               & UV              & Mostly blue   \\ \hline
%\end{tabular}
%}
%
%\frame{\frametitle{\textbf{Light bulbs}}
%\Large
%
%Why is one of these light bulbs so much easier to operate than the other?
%
%}

\frame{\frametitle{\textbf{Taking stock}}
	
	\Large
	\BC
	Where are we now?
	\EC
	
	\normalsize
	
	We understand:
	
	\BI
	\item How the things in the sky appear to move \pause
	\item How the things in the Solar System really move (and why)\pause
	\item How to study the spectra of things in the sky
	\item How to build telescopes to measure all of this
	\EI
	
	\BS\BS\pause
	
	Astronomy is off to the races! What will {\it we} study now?
	
	\BS
	
	{\bf In our time left, I want to focus on things that {\color{D}inform our humanity}.}
	\pause
	\BI
	\item \color{A} What's the history of our home, the Earth, and our neighbors? \pause
	\item \color{B} How might we affect it? \pause
	\item \color{C} How might we travel to other worlds? \pause
	\item \color{D} Could there be life there? 
	\EI
}
	
	


\frame{\frametitle{\textbf{The puzzle of the Sun's energy}}
	
	\Large \BC
	From lab this week, you know that the Sun produces an enormous amount of sunlight. \BS
	
	\EC
	
	\normalsize
	
	In our last unit, you encountered the idea of {\it conservation of energy:}
	
	\BI
	\item To produce light, an atom must lose energy\pause
	\item To absorb light, an atom must gain energy\pause
	\item {\color{B}Energy is never created or destroyed, only converted}\pause
	\EI
	
	\BS\BS
	
	
	\BC
	\Large
	\color{A}
	The Sun's thermal radiation converts its heat into light. \BS
	
	\color{C}
	But what energy source supplies that heat?
	\EC
}

\frame{\frametitle{\textbf{The contraction hypothesis}}
	
	\Large \BC
	In the 1800's, the best idea anyone had was that the Sun was collapsing slowly.
	\EC
	
	\normalsize
	
	\BI
	\item Water runs downhill and turns generators
	\item This uses {\it gravitational potential} as an energy source
	\item If the Sun is slowly falling inward, gravity could fuel it
	\pause
	\item {\color{A} ... for ten million years.}
    \EI

	\normalsize
	
	\BS
	\BI
	\item Geologists and biologists thought it had to be older
	\item Early 1900's: radioactive dating showed the Earth was {\color{C}definitely older}\pause
	\item \color{B}... we need a new idea to fuel the Sun!
	\EI
}

\frame{\frametitle{\textbf{New ideas in the 1920's}}

\BC
	Chemists noticed something weird in the masses of hydrogen and helium:
	
\BS

	\begin{tabular}{|c|c|c|}
		\hline
		& Mass                 & Energy ($E=mc^2$)                    \\ \hline
		One mole of hydrogen   & 1.008 grams          &                            \\ \hline
		Four moles of hydrogen & 4.032 grams          &                            \\ \hline
		One mole of helium     & 4.002 grams          &                            \\ \hline
		\textbf{Difference}    & \textbf{0.030 grams} & \textbf{3 trillion joules} \\ \hline
	\end{tabular}
\BS\BS
\EC

If hydrogen could be converted into helium, it would release an enormous amount of energy!

}
\frame{\frametitle{\textbf{The proposal of nuclear fusion}}
	\small
	\BCC
	\HC
	\BC
	\includegraphics[width=0.5\textwidth]{arthur-eddington.jpg}
	\EC
	In 1920 Arthur Eddington proposed that combining hydrogen into
	helium provided the energy source for the stars.
	
	\BI
	\item Only a little hydrogen would fuel the Sun for a long time
	\item Heavier elements could be made this way too (?)
	\EI
	
\pause
	\HC
	\BC
		\includegraphics[width=0.8\textwidth]{cecilia-payne.jpg}
		\EC
		
		In 1925 Cecilia Payne used spectroscopy to show (in her doctoral thesis!) 
		that the Sun was almost completely hydrogen.
		
		\BS\BS
		
		\color{A}This explains the source of the Sun's power!
	
	\ECC
		
}


\frame{\frametitle{\textbf{The Sun's history and the source of its power}}

\BC\includegraphics[width=0.5\textwidth]{hubble-formation.jpg}\\
\small (Hubble Space Telescope image: NASA + ESA / Judy Schmidt)

\BS

Clouds of gas -- mostly hydrogen but with a few heavier elements -- collapse
under their own gravity to form stars.

\EC
}

\frame{\frametitle{\textbf{The Sun's history and the source of its power}}
\Large
\BC
If you smash hydrogen nuclei together hard enough, they fuse to make helium
plus a {\it lot} of energy.

\BS

$$(P) + (P) + (P) + (P) \rightarrow (NNPP) + 2e^+$$

How much energy? We can calculate it...
\EC
}

\frame{\frametitle{\textbf{Nuclear potential energy}}

There is potential energy associated with the {\it arrangement of protons and neutrons in nuclei}.

\BS

We can calculate how much energy is associated with nuclear fusion by looking at how much potential energy
there is.

\BS

\begin{columns}
	\column{0.5\textwidth}
	\includegraphics[width=\textwidth]{binding.png}
		\column{0.5\textwidth}
%	\includegraphics[width=\textwidth]{binding-upsidedown.png}
\BI
\item Moving upward on this graph {\it releases} energy
\item Moving downward on this graph {\it requires} energy
\item Moving rightward combines smaller atoms to make bigger ones (fusion)
\item Moving leftward splits larger atoms to make smaller ones (fission)
\EI 
\end{columns}
}

\frame{

\BC

	\includegraphics[width=\textwidth]{binding.png}\EC}

\frame{\frametitle{\textbf{A star's life}}
	\BI
	\item Gravity compresses a star's core, heating it up
	\item Nuclear fusion starts, pushing back against gravity
	\item Once the nuclear fuel is depleted, gravity takes back over
	\item Once it reaches even higher temperatures, the next stage of fusion starts
	\EI
	
	
	\BS
	\BS
	
	\begin{tabular}{|c|c|c|}
		\hline
		& Energy produced           & Temperature required \\ 
		& \scriptsize(power plant time per ton) &     (Kelvin)         \\ \hline
		{\color{Red}Hydrogen to helium }         & \color{Red} 20 years                  &\color{Red} 10 million           \\ \hline
		{\color{Green}Helium to carbon  }          & \color{Green}2 years                    & \color{Green}100 million          \\ \hline
		Carbon to neon              & 1 year                    & 500 million          \\ \hline
		Neon to oxygen              & 5 days                 & 1 billion            \\ \hline
		Oxygen to silicon           & 1.5 years                    & 2 billion            \\ \hline
		Silicon to iron             & 10 months                    & 3 billion            \\ \hline\hline

		
		Uranium to fission products & 2-3 months                    & (Nuclear power)                     \\ \hline
		Coal   & 20 seconds        &                      \\ \hline

		Natural gas    & 45 seconds        &                      \\ \hline

	\end{tabular}
}
	

\frame{\frametitle{\textbf{The life of the Sun}}
	\BCC
	\HC
	\includegraphics[width=\textwidth]{hr.png}
	\footnotesize\it
	\BC
	(European Southern Observatory)
	\EC
	\HC
	Most stars are less massive than the Sun.
	
	\BS
	
	These ``red dwarfs'' lead long, cool, boring lives, slowly fusing hydrogen to helium, emitting red and infrared light. 
	
	\BS
	
	They are too faint for
	us to see without telescopes, but they contribute to the Milky Way glow. (Our nearest star is a red dwarf.)
	
	\BS
	
	They will live 10-100 times as long as the present age of the universe -- a trillion years.
	
	\BS 
	
	They will burn their hydrogen until it is all gone, then slowly fade away as brown dwarfs made of helium.
	
	\ECC
	
}

\frame{\frametitle{\textbf{The Sun's fate}}

\begin{columns}
\column{0.5\textwidth}
\normalsize

\BI
\item When the Sun runs out of hydrogen in its core, the core contracts,
while the outer layers puff up: it becomes a {\color{Red}red giant}. 
(5 billion years in the future, lasting for 1 billion years)

\item Eventually the core gets hot enough to fuse helium
into carbon, and the 
core ignites in a ``helium flash''. 

\item When the helium is depleted, that's it: the Sun isn't heavy enough
to fuse carbon.

\item The carbon core will be left behind as a white dwarf, slowly cooling
-- a dying ember in the sky, called a brown/black dwarf.

\item Its outer layers will be blown out into interstellar space,
briefly forming a nebula.

\EI

\column{0.5\textwidth}

\includegraphics[width=0.9\textwidth]{aldebaran-sun.png}
\footnotesize\it
\BC
(Wikimedia Commons)
\EC

\end{columns}
}


\frame{\frametitle{\textbf{Other stars}}
\BCC
\HC
\includegraphics[width=\textwidth]{onion-fusion.png}
\BC
\it \footnotesize Wikimedia Commons / R. J. Hall. \\Image not to scale.
\EC
\HC
\small
More massive stars have enough weight to compress their carbon cores and fuse it to (mostly) Ne, Na, Mg, and O. 

\BS

This process releases less energy than even helium fusion, so it doesn't last as long.

\BS

Elements fuse into heavier and heavier elements, releasing less energy each time, until they reach iron in the 
heaviest stars.

\BS

Iron is ``stellar ash'' -- it can't release any more energy by fusion.

\BS 

In some of these heaviest stars, once their iron cores grow too much, 
gravity crushes them into one enormous atomic nucleus -- a neutron star.

\ECC
}


\frame{\frametitle{\textbf{Supernovae}}
\BCC
\HC
\includegraphics[width=\textwidth]{crab-nebula.jpg}
\BC
\footnotesize\it (Hubble Space Telescope/NASA)
\EC
\HC
\small
The resulting explosion destroys the rest of the star.

\BS
\BS

It causes a flurry of nuclear reactions, forging elements heavier than iron.

\BS

It also scatters the heavy-element-rich contents of the star out to space.
This is why the Earth has so much iron in it -- and where our heavy elements come from.

\BS

It releases massive amounts of energy, forming a bright flash in the sky.

\BS

This is the Crab Nebula, the remnant of the 1054 supernova.

\BS

It was hundreds of times further away than most visible stars, but could be seen even during the day!

\ECC
}





\end{document}

