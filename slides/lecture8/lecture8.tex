
\documentclass[10pt]{beamer}
\usepackage{amsmath}
\usepackage{mathtools}
\usepackage{multimedia}
\usepackage{hyperref}


\usefonttheme{professionalfonts} % using non standard fonts for beamer
\usefonttheme{serif} % default family is serif
%\documentclass[12pt]{beamerthemeSam.sty}
\usepackage{epsf}
%\usepackage{pstricks}
%\usepackage[orientation=portrait,size=A4]{beamerposter}
\geometry{paperwidth=160mm,paperheight=120mm}
%DT favorite definitions
\def\LL{\left\langle}	% left angle bracket
\def\RR{\right\rangle}	% right angle bracket
\def\LP{\left(}		% left parenthesis
\def\RP{\right)}	% right parenthesis
\def\LB{\left\{}	% left curly bracket
\def\RB{\right\}}	% right curly bracket
\def\PAR#1#2{ {{\partial #1}\over{\partial #2}} }
\def\PARTWO#1#2{ {{\partial^2 #1}\over{\partial #2}^2} }
\def\PARTWOMIX#1#2#3{ {{\partial^2 #1}\over{\partial #2 \partial #3}} }

\def\rightpartial{{\overrightarrow\partial}}
\def\leftpartial{{\overleftarrow\partial}}
\def\diffpartial{\buildrel\leftrightarrow\over\partial}

\def\BC{\begin{center}}
\def\EC{\end{center}}
\def\BN{\begin{enumerate}}
\def\EN{\end{enumerate}}
\def\BI{\begin{itemize}}
\def\EI{\end{itemize}}
\def\BE{\begin{displaymath}}
\def\EE{\end{displaymath}}
\def\BEA{\begin{eqnarray*}}
\def\EEA{\end{eqnarray*}}
\def\BNEA{\begin{eqnarray}}
\def\ENEA{\end{eqnarray}}
\def\EL{\nonumber\\}

\newcommand{\etal}{{\it et al.}}
\newcommand{\gbeta}{6/g^2}
\newcommand{\la}[1]{\label{#1}}
\newcommand{\ie}{{\em i.e.\ }}
\newcommand{\eg}{{\em e.\,g.\ }}
\newcommand{\cf}{cf.\ }
\newcommand{\etc}{etc.\ }
\newcommand{\atantwo}{{\rm atan2}}
\newcommand{\Tr}{{\rm Tr}}
\newcommand{\dt}{\Delta t}
\newcommand{\op}{{\cal O}}
\newcommand{\msbar}{{\overline{\rm MS}}}
\def\chpt{\raise0.4ex\hbox{$\chi$}PT}
\def\schpt{S\raise0.4ex\hbox{$\chi$}PT}
\def\MeV{{\rm Me\!V}}
\def\GeV{{\rm Ge\!V}}

%AB: my color definitions
%\definecolor{mygarnet}{rgb}{0.445,0.184,0.215}
%\definecolor{mygold}{rgb}{0.848,0.848,0.098}
%\definecolor{myg2g}{rgb}{0.647,0.316,0.157}
\definecolor{A}{rgb}{1.0,0.3,0.3}
\definecolor{B}{rgb}{0.0,1.0,0.0}
\definecolor{C}{rgb}{1.0,1.0,0.0}
\definecolor{D}{rgb}{0.5,0.5,1.0}
\definecolor{E}{rgb}{0.7,0.7,0.7}
\definecolor{abtitlecolor}{rgb}{1.0,1.0,1.0}
\definecolor{absecondarycolor}{rgb}{0.0,0.416,0.804}
\definecolor{abprimarycolor}{rgb}{1.0,0.686,0.0}
\definecolor{Red}           {rgb}{1,0.4,0.4}
\definecolor{Yellow}           {rgb}{1,1,0.0}
\definecolor{Grey}          {cmyk}{.7,.7,.7,0}
\definecolor{Blue}          {cmyk}{1,1,0,0}
\definecolor{Green}         {cmyk}{1,0,1,0}
\definecolor{Brown}         {cmyk}{0,0.81,1,0.60}
\definecolor{Silver}        {rgb}{0.95,0.9,1.0}
\definecolor{Sky}           {rgb}{0.07,0.0,0.2}
\definecolor{Darkbrown}     {rgb}{0.4,0.3,0.2}
\definecolor{40Gray}        {rgb}{0.4,0.4,0.5}
\usetheme{Madrid}


\setbeamercolor{normal text}{fg=Silver,bg=Sky}

%AB: redefinition of beamer colors
%\setbeamercolor{palette tertiary}{fg=white,bg=mygarnet}
%\setbeamercolor{palette secondary}{fg=white,bg=myg2g}
%\setbeamercolor{palette primary}{fg=black,bg=mygold}
\setbeamercolor{title}{fg=abtitlecolor}
\setbeamercolor{frametitle}{fg=abtitlecolor}
\setbeamercolor{palette tertiary}{fg=white,bg=Darkbrown}
\setbeamercolor{palette secondary}{fg=white,bg=absecondarycolor}
\setbeamercolor{palette primary}{fg=white,bg=40Gray}
\setbeamercolor{structure}{fg=abtitlecolor}

\setbeamerfont{section in toc}{series=\bfseries}

%AB: remove navigation icons
\beamertemplatenavigationsymbolsempty
\title[A few words on the philosophy of science]
  \textbf {A few words on the philosophy of science}


\author [Astronomy 101]{Astronomy 101\\Syracuse University, Fall 2016\\Walter Freeman}

\date{\today}

\begin{document}



\frame{\titlepage}

\frame{\frametitle{\textbf{Announcements}}
\Large
\BI
\item{Walter is still on vocal rest and won't be speaking today}
\item{Several of our TA's will lead class today}
\item{The first writing assignment has been posted}
\item{Exams returned in lab this week}
\item{Take-home lab posted, due early December}
\EI
}

\frame{\frametitle{\textbf{Writing assignment}}
\Large \BC The full thing is on the website. In brief:\EC

\BI
\item Choose a historical calendar
\item Research it
\item Write one page (or more) on how it describes the motion of the sky
\item Due in two weeks
\item Potential for significant extra credit
\item Some special assignments for particular calendars; read the whole thing

}

\frame{\frametitle{\textbf{Exams}}

(to be filled in once I hear back from Scott)

}

\frame{\frametitle{\textbf{Take-home lab}}
\Large
\BC This is also posted, and is due early December.

\bigskip
\bigskip
\bigskip

{\bf Start early} -- the observations span several weeks.

}



\frame{\frametitle{\textbf{What have we done so far?}}
\Large
\BC
We are now able to predict the motions of most of the stuff in the night sky:
\EC
\BI
\item the distant stars
\item the Sun
\item the Moon
\item {\bf not} the planets!
\EI
}

\frame{\frametitle{\textbf{We've casually mixed together ancient and modern perspectives}}

\begin{columns}
\column{0.5\textwidth}
\Large
\BC The celestial sphere model \EC
\column{0.5\textwidth}
\Large
\BC The Earth-orbits-Sun model \EC
\end{columns}

\begin{columns}
\column{0.5\textwidth}
\large
\BI
\item{Heavenly bodies stuck to spheres}
\item{Spheres all turn around Earth}
\item{Planets, Sun, and Moon all have their own spheres}
\item{``Epicycles'' needed to get planets right}
\EI
\BC The celestial sphere model \EC
\column{0.5\textwidth}
\Large
\BC The Earth-orbits-Sun model \EC
\BI
\item{Earth is one of many planets, all orbiting the Sun}
\item{The Earth rotates on its axis}
\item{The stars are very far away and don't move}
\item{Modern perspective}
\EI
\end{columns}



\frame{

\Huge \BC Complete {\it Lecture Tutorials} pp. 85-88.\EC

}

\frame{

\Huge

When the waxing half moon is just rising over the horizon, it is closest to:

\bigskip
\bigskip
\bigskip

\color{A}A: 6AM \\
\color{B}B: Noon \\
\color{C}C: 6PM \\
\color{D}D: Midnight
}

\frame{

\Huge

As seen in the Northern Hemisphere, which part of a waning crescent moon will be lit?

\bigskip
\bigskip
\bigskip

\color{A}A: The right part \\
\color{B}B: The left part \\
}

\frame{\frametitle{\textbf{Oddities in the sky}}

\Large

So far we've talked about the Sun, the Moon, and the stars. What about...

\pause

\BI
\item{planets}
\pause
\item{comets}
\pause
\item{eclipses}
\pause
\item{meteors}
\pause
\item{...}
\EI
}

\frame{\frametitle{\textbf{The planets: what has gone wrong?}}

\Large
\BC
Demo on {\it Stellarium}
\EC

\pause
\bigskip


Sometimes some planets appear to go backwards (``retrograde motion''). 

\bigskip
\bigskip

\large

This tells us that celestial sphere model can't be literally true. Why does it work for everything else?

\BI
\item{The celestial sphere model works if things appear to only rotate around the Earth.}
\item{The stars are so far away that only the Earth's rotation matters}
\item{The Earth orbits the Sun, so we just pretend that the Sun is on a different sphere turning a bit slower, taking into account both our revolution around it and our rotation}
\item{The Moon orbits the Earth, so we again put the Moon on a different sphere, turning slower}
\item{... but how can we get a sphere to go forwards and backwards?}
\item{\bf The celestial sphere model gets the motion of the planets badly wrong}
\EI
}

\frame{\frametitle{\textbf{Oh, you sweet summer child...}}
\Large
\BC
Why are the changes in the seasons in {\it Game of Thrones} so terrifying?
\EC

\pause 
\bigskip
\bigskip
\bigskip

... they're unpredictable!

\bigskip

We've long used the immutability of the sky as a symbol for constancy. The cycles of the Sun, Moon, and stars don't ever change, but some things do!

\bigskip

These unexpected things in the sky once terrified people; now we know why they happen.

}

\frame{\frametitle{\textbf{Eclipses}}
\Large
You know that during a new moon, the Moon lies roughly between the Earth and the Sun.

\bigskip
\bigskip
\bigskip

However, the Moon's orbit is tilted just a bit, so it usually passes over or under the Sun.

\bigskip

\BC \includegraphics[width=0.27\textwidth]{solar-eclipse.jpg}\EC

\bigskip

If it passes in front, you get a solar eclipse!

This terrified many of the ancients -- ``the Sun got eaten! We're doomed!''

}


\frame{\frametitle{\textbf{Eclipses}}
\Large
You know that during a full moon, the Earth lies roughly between the Moon and the Sun.

\bigskip
\bigskip
\bigskip

Same deal: usually the Earth's shadow misses the Moon. Sometimes (like last weekend!) it doesn't!

\bigskip

\BC \includegraphics[width=0.27\textwidth]{lunar-eclipse.jpg}\EC

\bigskip

\large

Here some light is refracted by the atmosphere. The blue component is scattered away by the atmosphere; the red component bends and hits the Moon.

}

\frame{\frametitle{\textbf{Meteors}}

\large

Orbits of things in the Solar System are not always close to circular.

\bigskip

There are lots of small things in the Solar System, many of which have elongated orbits that sometimes cross ours.

\bigskip

Meteors:

\large

\begin{columns}
\column{0.6\textwidth}

\BI
\item{Little rocky or metallic bits of matter that orbit the Sun}
\item{Sometimes they get to Earth and glow as atmospheric drag heats them}
\item{Sometimes they hit the surface, and we get chunks of space-slag}
\item{Historical cultures sometimes used them as easy access to metal}
\EI
\column{0.4\textwidth}
\BC\includegraphics[width=\textwidth]{meteorite.jpg}\EC
\end{columns}
}

\frame{\frametitle{\textbf{Comets}}

\Large

Comets are ``dirty snowballs'' whose orbits are {\it highly} elongated.

\large

\begin{columns}
\column{0.6\textwidth}

\BI
\item{Mostly made of ice}
\item{When they get close to the Sun, the heat melts bits off of them}
\item{This stream of stuff reflects sunlight and makes the comet's ``tail''}
\item{Historical cultures were often terrified of them, but they're just space-snowballs}
\EI
\column{0.4\textwidth}
\BC\includegraphics[width=\textwidth]{comet-rosetta.jpg}\EC
\end{columns}
}

\frame{\frametitle{\textbf{The exam}}

\Huge
\BI
\item{Around 30 multiple choice questions}
\item{Three short-answer questions}
\item{All you need is a pencil}
\EI
}

\frame{\frametitle{\textbf{The exam: what to study}}
\Large

The exam covers, in descending order of emphasis:


\BI
\item{The material in the {\it Lecture Tutorials}}
\item{The material in the homework and labs}
\item{The material we talked about in class that was {\it not} in the Lecture Tutorials, including demos, videos, etc.}
\item{The material in the textbook}
\EI

}

\frame{

\BC\Huge Any questions?\EC

}

\end{document}

