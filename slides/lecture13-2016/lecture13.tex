
\documentclass[10pt]{beamer}
\usepackage{amsmath}
\usepackage{mathtools}
\usepackage{multimedia}
\usepackage{hyperref}


\usefonttheme{professionalfonts} % using non standard fonts for beamer
\usefonttheme{serif} % default family is serif
%\documentclass[12pt]{beamerthemeSam.sty}
\usepackage{epsf}
%\usepackage{pstricks}
%\usepackage[orientation=portrait,size=A4]{beamerposter}
\geometry{paperwidth=160mm,paperheight=120mm}
%DT favorite definitions
\def\LL{\left\langle}	% left angle bracket
\def\RR{\right\rangle}	% right angle bracket
\def\LP{\left(}		% left parenthesis
\def\RP{\right)}	% right parenthesis
\def\LB{\left\{}	% left curly bracket
\def\RB{\right\}}	% right curly bracket
\def\PAR#1#2{ {{\partial #1}\over{\partial #2}} }
\def\PARTWO#1#2{ {{\partial^2 #1}\over{\partial #2}^2} }
\def\PARTWOMIX#1#2#3{ {{\partial^2 #1}\over{\partial #2 \partial #3}} }

\def\rightpartial{{\overrightarrow\partial}}
\def\leftpartial{{\overleftarrow\partial}}
\def\diffpartial{\buildrel\leftrightarrow\over\partial}

\def\BC{\begin{center}}
\def\EC{\end{center}}
\def\BN{\begin{enumerate}}
\def\EN{\end{enumerate}}
\def\BI{\begin{itemize}}
\def\EI{\end{itemize}}
\def\BE{\begin{displaymath}}
\def\EE{\end{displaymath}}
\def\BEA{\begin{eqnarray*}}
\def\EEA{\end{eqnarray*}}
\def\BNEA{\begin{eqnarray}}
\def\ENEA{\end{eqnarray}}
\def\EL{\nonumber\\}

\newcommand{\etal}{{\it et al.}}
\newcommand{\gbeta}{6/g^2}
\newcommand{\la}[1]{\label{#1}}
\newcommand{\ie}{{\em i.e.\ }}
\newcommand{\eg}{{\em e.\,g.\ }}
\newcommand{\cf}{cf.\ }
\newcommand{\etc}{etc.\ }
\newcommand{\atantwo}{{\rm atan2}}
\newcommand{\Tr}{{\rm Tr}}
\newcommand{\dt}{\Delta t}
\newcommand{\op}{{\cal O}}
\newcommand{\msbar}{{\overline{\rm MS}}}
\def\chpt{\raise0.4ex\hbox{$\chi$}PT}
\def\schpt{S\raise0.4ex\hbox{$\chi$}PT}
\def\MeV{{\rm Me\!V}}
\def\GeV{{\rm Ge\!V}}

%AB: my color definitions
%\definecolor{mygarnet}{rgb}{0.445,0.184,0.215}
%\definecolor{mygold}{rgb}{0.848,0.848,0.098}
%\definecolor{myg2g}{rgb}{0.647,0.316,0.157}
\definecolor{A}{rgb}{1.0,0.3,0.3}
\definecolor{B}{rgb}{0.0,1.0,0.0}
\definecolor{C}{rgb}{1.0,1.0,0.0}
\definecolor{D}{rgb}{0.5,0.5,1.0}
\definecolor{E}{rgb}{0.7,0.7,0.7}
\definecolor{abtitlecolor}{rgb}{1.0,1.0,1.0}
\definecolor{absecondarycolor}{rgb}{0.0,0.416,0.804}
\definecolor{abprimarycolor}{rgb}{1.0,0.686,0.0}
\definecolor{Red}           {rgb}{1,0.4,0.4}
\definecolor{Yellow}           {rgb}{1,1,0.0}
\definecolor{Grey}          {cmyk}{.7,.7,.7,0}
\definecolor{Blue}          {cmyk}{1,1,0,0}
\definecolor{Green}         {cmyk}{1,0,1,0}
\definecolor{Brown}         {cmyk}{0,0.81,1,0.60}
\definecolor{Silver}        {rgb}{0.95,0.9,1.0}
\definecolor{Sky}           {rgb}{0.07,0.0,0.2}
\definecolor{Darkbrown}     {rgb}{0.4,0.3,0.2}
\definecolor{40Gray}        {rgb}{0.4,0.4,0.5}
\usetheme{Madrid}


\setbeamercolor{normal text}{fg=Silver,bg=Sky}

%AB: redefinition of beamer colors
%\setbeamercolor{palette tertiary}{fg=white,bg=mygarnet}
%\setbeamercolor{palette secondary}{fg=white,bg=myg2g}
%\setbeamercolor{palette primary}{fg=black,bg=mygold}
\setbeamercolor{title}{fg=abtitlecolor}
\setbeamercolor{frametitle}{fg=abtitlecolor}
\setbeamercolor{palette tertiary}{fg=white,bg=Darkbrown}
\setbeamercolor{palette secondary}{fg=white,bg=absecondarycolor}
\setbeamercolor{palette primary}{fg=white,bg=40Gray}
\setbeamercolor{structure}{fg=abtitlecolor}

\setbeamerfont{section in toc}{series=\bfseries}

%AB: remove navigation icons
\beamertemplatenavigationsymbolsempty
\title[Celestial mechanics]{
  \textbf {Celestial mechanics}}


\author [Astronomy 101]{Astronomy 101\\Syracuse University, Fall 2016\\Walter Freeman}

\date{\today}

\begin{document}



\frame{\titlepage}


\frame{\frametitle{\textbf{Announcements}}
\Large
\BI
\item{Per email: blanket mini-extension given to everyone for Mastering Astronomy until tonight at midnight}
\item{Study guide posted}
\item{Exam prep schedule:}
\BI
\item{Thursday: Class as usual, with a mini-review and time for questions}
\item{Friday: Help session Friday morning (9:30-11:30) and afternoon (1:30-3:30)}
\item{Sunday: Review session, times announced Thursday}
\item{Monday: Available all day for questions.}
\item{Extra coach in Clinic: Friday at 4:45 PM - 7:45 PM and Monday at 2:05 PM ~ 5:05PM}
\EI
\item Remember, there are astronomy coaches and/or physics PhD students in the Clinic all the time to help you!
\EI
}

\frame{\frametitle{\textbf{For the exam:}}

\BC
\Large
Bring your student ID (or memorize your SUID)!

\bigskip
\bigskip
\bigskip

\large

The biggest cause of grading issues is {\it SUID errors}. Reducing these will help everything go more smoothly.

\EC

}

\frame{\frametitle{\textbf{Last time}}
\Large

Newton's laws of motion (the first two):

\large

\BI
\item{Objects continue moving in a straight line at a constant velocity unless a force acts on them}
\item{Forces make objects accelerate}
\EI

The terms used here have particular meanings:

\BI
\item{{\bf Velocity:} The speed {\it and direction} of an object's motion}
\item{{\bf Acceleration:} A change in an object's velocity (speeding up, slowing down, or changing direction}
\EI

\bigskip
\bigskip

In symbols, Newton's second law can be written:

$$(\rm object's\,acceleration) = (\rm force\,applied\,to\,object) \div (\rm object's\,mass)$$

or

$$F=ma.$$
}

\frame{\frametitle{\textbf{Last time: gravity}}
\Large
We also saw that

$$
F_{\rm grav} = \frac{G \times ({\rm mass\, of\, object\, A}) \times ({\rm mass\, of\, object\, B})}{({\rm distance\, between\, them})^2}
$$

\pause\bigskip\bigskip

or, in mathematical shorthand,

$$
F_{\rm grav} = \frac{Gm_1m_2}{r^2}.
$$


\large
\pause

\bigskip
\bigskip

$G$ is just a number telling how strong gravity is: about a ten-billionth of the weight of an apple for two kilogram objects a meter apart.
} 

\frame{

\BC
\Large
Now we're going to see how Newton's discovery of the principles of physics explains some features of Kepler's laws.
\EC
}


\frame{\frametitle{\textbf{How does gravity make things accelerate?}}

\BC
\large Find the acceleration due to gravity using \\Newton's second law $F=ma$ and $F_{\rm grav}=\frac{GMm}{r^2}$.
\EC

\bigskip
\bigskip

\normalsize

Here I'm using $m$ for the object being accelerated, and $M$ for the mass of the other object. 

\bigskip

If we're looking at how Earth's causes a satellite to accelerate in its orbit around it, $M$ is the mass of the Earth and $m$ is the mass of the satellite.

\bigskip
\bigskip
\bigskip
\Large
\begin{columns}
\column{0.5\textwidth}
$$F_g=ma$$
\column{0.5\textwidth}
$$F_g = \frac{GMm}{r^2}$$
\end{columns}

\BC
Set the two equal:
\EC
$${\color{Red} m}a = \frac{GM{\color{Red}m}}{r^2}$$
}

\frame{\frametitle{\textbf{How does gravity make things accelerate?}}
\Large
\begin{columns}
\column{0.5\textwidth}
$$F_g=ma$$
\column{0.5\textwidth}
$$F_g = \frac{GMm}{r^2}$$
\end{columns}

\BC
Set the two equal:
\EC
$${\color{Red} m}a = \frac{GM{\color{Red}m}}{r^2}$$
\BC
The $m$'s (mass of the satellite, here) cancel!
\EC
$$a = \frac{GM}{r^2}$$

\large
\BC
The acceleration {\color{Red}doesn't depend on mass!} 
\EC
\bigskip

This explains why Kepler's third law, about orbital times, doesn't depend on the mass of the planet; the acceleration from gravity doesn't depend on the 
mass being accelerated.
}



\frame{\frametitle{\textbf{How does this create circular motion?}}
\Large
Without a force, things travel in straight lines at constant speeds (Newton's first law).

\bigskip

It requires a force {\it directed toward the center} to hold something in circular motion.

\bigskip
\bigskip

Let's demonstrate and watch this.}


\frame{\frametitle{\textbf{Which way will the ball go?}}
\Large
I'm going to whirl a ball around a circular metal track. When I remove a bit of the track, the ball will fly out.
Which way will it go? (Answer choices on document camera)}


\frame{\frametitle{\textbf{How does this create circular motion?}}
\Large

Consider a planet orbiting a star:

\BI
\item Gravity always pulls the planet inward
\item A force directed toward the center is required for a circular orbit
\item Very specific balance between orbital speed and distance required for a circle (otherwise we get an ellipse!)
\EI
}

\frame{
\Large

\BC Let's watch this. Note: \EC

\BI
\item The planet's {\it velocity} {\color{Red} (red arrow)} is always pointed in the direction it is moving
\item The planet's {\it acceleration} {\color{Green} (green arrow)} is always pointed toward the Sun
\EI
}



\frame{\frametitle{\textbf{What about elliptical orbits?}}

\Large
For gravity, the force depends on the distance from the center, as you know. 

\bigskip

The particular mathematics that produces ellipses is beyond the scope of this class. But we can understand the principles!}


\frame{\frametitle{\textbf{What about elliptical orbits?}}
\large
``If we imagine a line connecting a comet and the sun, we can see that when a comet is closer to the sun, the line is smaller and when the comet is further, the line is longer. Newton's law of gravity says that the shorter the line gets, the stronger the connection (gravity) is between the sun and the comet. Conversely, the longer the line gets, the weaker the connection (gravity) is between the comet and the sun.''

\bigskip
\bigskip

\begin{flushright}--Ariana\end{flushright}
}

\frame{\frametitle{\textbf{What about elliptical orbits?}}
\Large
If the comet is closer to the Sun, that means:

\BI
\item The gravitational force is higher
\item The acceleration is higher
\item The comet's velocity changes direction faster
\item It ``whips around the Sun'' near perihelion (point of close approach)
\EI

\bigskip
\bigskip

When it is further from the Sun:

\BI
\item The gravitational force is very small
\item It takes a long time to change direction
\item The Sun's gravity takes a long time to arrest the comet's motion and turn it back around
\EI
}

\frame{\frametitle{\textbf{Angular momentum}}
\Large
Another consequence of the mathematics of Newton's laws is {\it angular momentum}.

\bigskip
\bigskip

Any spinning or revolving object has angular momentum.

\bigskip
\bigskip

Angular momentum = (mass) $\times$ (how far it is from the center of motion) $\times$ (how fast it is moving around the center)

\bigskip
\bigskip

Angular momentum is {\color{Red}conserved} -- unless an outside agent messes with it, this product will always remain the same!

\pause

\bigskip
\bigskip

\url{https://www.youtube.com/watch?v=VmeM0BNnGR0}

}

\frame{\frametitle{\textbf{Angular momentum and Kepler's laws}}
\Large
Consider an elliptical orbit like the one displayed. When the planet is closer to the Sun, is its angular momentum:

\bigskip
\bigskip

\color{A}A: Larger than when it is far from the Sun\\
\color{B}B: Smaller than when it is far from the Sun\\
\color{C}C: The same as when it is far from the Sun
}

\frame{\frametitle{\textbf{Angular momentum and Kepler's laws}}
\Large
Consider an elliptical orbit like the one displayed. When the planet is closer to the Sun, is its speed around the Sun

\bigskip
\bigskip

\color{A}A: Larger than when it is far from the Sun\\
\color{B}B: Smaller than when it is far from the Sun\\
\color{C}C: The same as when it is far from the Sun
}

\frame{\frametitle{\textbf{Angular momentum and Kepler's laws}}

\Large

Angular momentum = (mass) $\times$ {\color{Red}(how far it is from the Sun)} $\times$ {\color{Green}(how fast it is moving around the Sun)}

\bigskip
\bigskip
\bigskip

\pause

Remember, angular momentum is {\color{Red}conserved} -- this product is always the same. 

\bigskip

If the distance from the Sun increases, the speed it moves around the Sun must decrease to keep angular momentum the same.

\pause

\bigskip
\bigskip
\bf Kepler's second law is a {\color{Red}consequence} of the conservation of angular momentum!
}

\frame{\frametitle{\textbf{The full machinery of Newton}}

\Large
$F=ma$ along with Newton's law of gravity is enough to calculate that orbits are ellipses, and all of their properties.

\bigskip

The math here is hard if you don't have a computer to help you.
\large

\bigskip


\BI
\item{I've shown you simulations with a few planets}
\item{My poor dying laptop can handle perhaps a hundred planets before it gets unbearably slow}
\item{... a supercomputer, with faster algorithms than I have, can do very many more!}
\pause
\item{Remember this? \url{https://youtu.be/PrIk6dKcdoU}}
\item{This is all a consequence of Newton's laws, just with many more objects than I've used}
\EI
}


\frame{\frametitle{\textbf{The clockwork universe}}
``Our imagination only beautifies the World surrounding us; however, the successes of physics help us know more about the "truth" in this world. Moreover, when the word "World" means some place bigger than the Earth, the successful physics brings us to a new system - the Universe, which is much more attractive and special.''

\bigskip
\bigskip

\begin{flushright}
--Boting Zhu
\end{flushright}


}

\frame{

``I think that this "clockwork universe" picture detracts from the romance of the world around us. For example, when you look at a complex math equation with 6 different variables and 10 different operations between them, do we marvel at its beauty? No, we get a headache and we get confused. I believe that there is beauty in simplicity, and thats why I do not look at Nature as an immensely complicated machine. Our successes of physics are helping our quest to find beauty and meaning in the world around us because even though the mathematics behind the new models and theories may be difficult, the underlying ideas are helping to explain the simplicity of the world. For example, even though we are learning that gravity has many complexities, gravity is simply just a force. There is beauty in that simplicity.''

\begin{flushright}
--Lillee
\end{flushright}

\pause

``I believe that the clockwork universe is inspiring, and while it is necessary to see how it works scientifically, it would be quite a beautiful thing to believe that the universe operates so simply, and isn't as complex as it's made out to be. The successes of physics detract from the natural beauty of it all, but are integral towards our understanding of how everything works.''

\pause

\begin{flushright}
--``Ken Bone Apple Tea'', apparently a memelord or a Redditor
\end{flushright}
}

\frame{
``I think the clockwork nature of our universe is romantic, it doesn't take away from the beauty of the world around us, it contributes to it. To believe the universe is itself far more connected than most people can imagine is an idea which contains its own complex beauty. The successes of physics will allow us to develop a more in depth understanding of the universe in which we reside, which can lead to a relationship in which all parts of the universe, big and small, are seen as central to our lives.''

\begin{flushright}
--``Ich bin ein Berliner'', apparently either a president or a jelly doughnut
\end{flushright}
}

\frame{
``I believe that every time we keep discovering more about our universe, it keeps becoming more complicated, and it keeps becoming more beautiful in our eyes. The universe is something very complicated to understand, but really easy to enjoy. I think the "clockwork universe" picture is very inspiring, and it shows us the beauty of the world. The successes of physics mean that everyday we are going to know more and more about the universe, which also means we are going to enjoy the beauty of it more and more as the years go by.''
\begin{flushright}
--Eduardo Sola
\end{flushright}
}

\frame{
``[T]he imagination of nature is far, far greater than the imagination of man (sic). For instance, how much more remarkable it is for us to be stuck -- half of us upside down -- by a mysterious attraction, to a spinning ball that has been swinging in space for billions of years, than to be carried on the back of an elephant supported on a tortoise swimming in a bottomless sea.

\bigskip

For instance, I stand at the seashore, alone, and start to think. 

\smallskip

There are the rushing waves, mountains of molecules, each stupidly minding its own business, trillions apart, yet forming white surf in unison.

\smallskip


Ages on ages, before any eyes could see, year after year, thunderously pounding the shore as now. 

\smallskip


For whom, for what? On a dead planet, with no life to entertain.

\smallskip


Never at rest,  tortured by energy, wasted prodigiously by the sun, poured into space. A mite makes the sea roar.


\smallskip

Deep in the sea, all molecules repeat the patterns of one another till complex new ones are formed. They make others like themselves, and a new dance starts.

\smallskip


Growing in size and complexity: living things, masses of atoms, DNA, protein... dancing a pattern ever more intricate.

\medskip


Out of the cradle onto the dry land, here it is standing: atoms with consciousness,  matter with curiosity.

\smallskip


Stands at the sea, wonders at wondering: I, a universe of atoms, an atom in the universe.''

\begin{flushright}
--Richard Feynman (again), from {\it The Value of Science} (1955)
\end{flushright}
}

\end{document}
