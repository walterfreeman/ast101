
\documentclass[10pt]{beamer}
\usepackage{amsmath}
\usepackage{mathtools}
\usepackage{multimedia}
\usepackage{hyperref}


\usefonttheme{professionalfonts} % using non standard fonts for beamer
\usefonttheme{serif} % default family is serif
%\documentclass[12pt]{beamerthemeSam.sty}
\usepackage{epsf}
%\usepackage{pstricks}
%\usepackage[orientation=portrait,size=A4]{beamerposter}
\geometry{paperwidth=160mm,paperheight=120mm}
%DT favorite definitions
\def\LL{\left\langle}	% left angle bracket
\def\RR{\right\rangle}	% right angle bracket
\def\LP{\left(}		% left parenthesis
\def\RP{\right)}	% right parenthesis
\def\LB{\left\{}	% left curly bracket
\def\RB{\right\}}	% right curly bracket
\def\PAR#1#2{ {{\partial #1}\over{\partial #2}} }
\def\PARTWO#1#2{ {{\partial^2 #1}\over{\partial #2}^2} }
\def\PARTWOMIX#1#2#3{ {{\partial^2 #1}\over{\partial #2 \partial #3}} }

\def\rightpartial{{\overrightarrow\partial}}
\def\leftpartial{{\overleftarrow\partial}}
\def\diffpartial{\buildrel\leftrightarrow\over\partial}

\def\BC{\begin{center}}
\def\EC{\end{center}}
\def\BN{\begin{enumerate}}
\def\EN{\end{enumerate}}
\def\BI{\begin{itemize}}
\def\EI{\end{itemize}}
\def\BE{\begin{displaymath}}
\def\EE{\end{displaymath}}
\def\BEA{\begin{eqnarray*}}
\def\EEA{\end{eqnarray*}}
\def\BNEA{\begin{eqnarray}}
\def\ENEA{\end{eqnarray}}
\def\EL{\nonumber\\}

\newcommand{\etal}{{\it et al.}}
\newcommand{\gbeta}{6/g^2}
\newcommand{\la}[1]{\label{#1}}
\newcommand{\ie}{{\em i.e.\ }}
\newcommand{\eg}{{\em e.\,g.\ }}
\newcommand{\cf}{cf.\ }
\newcommand{\etc}{etc.\ }
\newcommand{\atantwo}{{\rm atan2}}
\newcommand{\Tr}{{\rm Tr}}
\newcommand{\dt}{\Delta t}
\newcommand{\op}{{\cal O}}
\newcommand{\msbar}{{\overline{\rm MS}}}
\def\chpt{\raise0.4ex\hbox{$\chi$}PT}
\def\schpt{S\raise0.4ex\hbox{$\chi$}PT}
\def\MeV{{\rm Me\!V}}
\def\GeV{{\rm Ge\!V}}

%AB: my color definitions
%\definecolor{mygarnet}{rgb}{0.445,0.184,0.215}
%\definecolor{mygold}{rgb}{0.848,0.848,0.098}
%\definecolor{myg2g}{rgb}{0.647,0.316,0.157}
\definecolor{A}{rgb}{1.0,0.3,0.3}
\definecolor{B}{rgb}{0.0,1.0,0.0}
\definecolor{C}{rgb}{1.0,1.0,0.0}
\definecolor{D}{rgb}{0.5,0.5,1.0}
\definecolor{E}{rgb}{0.7,0.7,0.7}
\definecolor{abtitlecolor}{rgb}{1.0,1.0,1.0}
\definecolor{absecondarycolor}{rgb}{0.0,0.416,0.804}
\definecolor{abprimarycolor}{rgb}{1.0,0.686,0.0}
\definecolor{Red}           {rgb}{1,0.4,0.4}
\definecolor{Yellow}           {rgb}{1,1,0.0}
\definecolor{Grey}          {cmyk}{.7,.7,.7,0}
\definecolor{Blue}          {cmyk}{1,1,0,0}
\definecolor{Green}         {cmyk}{1,0,1,0}
\definecolor{Brown}         {cmyk}{0,0.81,1,0.60}
\definecolor{Silver}        {rgb}{0.95,0.9,1.0}
\definecolor{Sky}           {rgb}{0.07,0.0,0.2}
\definecolor{Darkbrown}     {rgb}{0.4,0.3,0.2}
\definecolor{40Gray}        {rgb}{0.4,0.4,0.5}
\usetheme{Madrid}


\setbeamercolor{normal text}{fg=Silver,bg=Sky}

%AB: redefinition of beamer colors
%\setbeamercolor{palette tertiary}{fg=white,bg=mygarnet}
%\setbeamercolor{palette secondary}{fg=white,bg=myg2g}
%\setbeamercolor{palette primary}{fg=black,bg=mygold}
\setbeamercolor{title}{fg=abtitlecolor}
\setbeamercolor{frametitle}{fg=abtitlecolor}
\setbeamercolor{palette tertiary}{fg=white,bg=Darkbrown}
\setbeamercolor{palette secondary}{fg=white,bg=absecondarycolor}
\setbeamercolor{palette primary}{fg=white,bg=40Gray}
\setbeamercolor{structure}{fg=abtitlecolor}

\setbeamerfont{section in toc}{series=\bfseries}

%AB: remove navigation icons
\beamertemplatenavigationsymbolsempty
\title[The beginnings of physics]{
  \textbf {The beginnings of physics}}


\author [Astronomy 101]{Astronomy 101\\Syracuse University, Fall 2018\\Walter Freeman}

\date{\today}

\begin{document}



\frame{\titlepage}

%\frame{
%
%
%"What, then, is the meaning of it all? ... I think that we must frankly admit that {\em we do not know.}
%
%\bigskip
%
%This is not a new idea; this is the idea of the age of reason.  This is the philosophy that guided the men who made the democracy that we live under. 
%The idea that no one really knew how to run a government led to the idea that we should arrange a system by which new ideas could be developed, tried out, tossed out, more new ideas brought in; a trial and error system.  
%This method was a result of the fact that science was already showing itself to be a successful venture at the end of the 18th century.  
%Even then it was clear to socially‑minded people that the openness of the possibilities was an opportunity, and that doubt and discussion were essential to progress into the unknown.  
%If we want to solve a problem that we have never solved before, we must leave the door to the unknown ajar.
%
%\bigskip
%
%We are at the very beginning of time for the human race. It is not unreasonable that we grapple with problems....  Our responsibility is to do what we can, learn what we can, improve the solutions and pass them on.  
%It is our responsibility to leave the men of the future a free hand.  In the impetuous youth of humanity, we can make grave errors that can stunt our growth for a long time.
%This we will do if we say we have the answers now, so young and ignorant; if we suppress all discussion, all criticism, saying, "This is it, boys, man is saved!" and thus doom man for a long time to the chains of authority, confined to the limits of our present imagination.  It has been done so many times before.
%
%\bigskip
%
%It is our responsibility as scientists, knowing the great progress and great value of a satisfactory philosophy
%of ignorance, the great progress that is the fruit of freedom of thought, to proclaim the value of this freedom, 
%to teach how doubt is not to be feared but welcomed and discussed, and to demand this freedom as our duty to all coming generations.
%
%\bigskip
%
%\large
%\begin{flushright}--Feynman, {\it The Value of Science} (1955)\end{flushright}
%}
%
%

\frame{\frametitle{\textbf{Announcements}}
\Large
\BI
\item No office hours this week (I've been invited to give a seminar in DC)
\item I fixed a bunch of grade issues (ODS people; SUID errors)
\item Grades for makeup exams still pending (sorry)
\EI
}

\frame{\frametitle{\textbf{Exam 2 approaches}}
\Large
\BC Exam 2 is next Tuesday. Reminders:\EC

\bigskip
\BI
\item You may bring a single-sided page of notes, as before
\item You won't need your globes (nor will they help)
\item I'll be holding an extra review on Monday from 2PM-5PM
\item The study guide for Exam 2 is posted
\EI
}



\frame{\frametitle{\textbf{On the papers}}

\Large

We've started to grade your papers.

\bigskip
\bigskip


Issue 1: Don't just list facts, tell why they are important and how they work!

\bigskip
\bigskip


{\large\color{Red}
\it The Jewish calendar has 12 months in a year. Each month has 29 or 30 days. Every few years there is a
thirteenth month added; this is called an intercalary month.
}

\begin{center} ... compared with ... \end{center}


{\large\color{Green}
\it The Jewish calendar has 12 months in a year. These months are exactly based on the phases of the moon,
unlike the months in our calendar, which are only approximate. But 12 lunar cycles is about ten days short
of a full year. This means that around every third year, a 13th month must be added to the calendar, so that
the New Year happens in the same season each time.}

}


\frame{\frametitle{\textbf{On the papers}}

\Large

We've started to grade your papers.

\bigskip
\bigskip


Issue 2: If you don't understand something (words, ideas), ask about it, rather than glossing over it!

\bigskip
\bigskip


{\large\color{Red}
\it The ancient Egyptian year begins with the {\color{D}heliacal} rising of Sirius. 
}

\begin{center} ... compared with ... \end{center}


{\large\color{Green}
\it Since the sidereal day and solar day are not equal, stars rise at different hours of the day in different 
parts of the year. When the Sun has already risen, you can't see the stars. This means that during part of the
year, we won't see Sirius rise. However, since Sirius will rise earlier and earlier each day, eventually it 
will come up before the Sun has risen, and you'll see Sirius rise again. This event marked the ancient Egyptian
new year.}

}

\frame{\frametitle{\textbf{On the papers}}

\Large

We've started to grade your papers.

\bigskip
\bigskip


Issue 3: Fluff 

\bigskip
\bigskip


{\large\color{Red}
\it The Flubowilli calendar is very interesting! It's very different from our own Gregorian calendar! 
}

\begin{center} ... compared with ... \end{center}


{\large\color{Green}
\it The Flubowilli people live on a planet near two stars that orbit each other. This gives them the ability 
to tell time by the motion of their suns, which is the basis of their calendar.
}
}




\frame{\frametitle{\textbf{Newton's three ideas, one at a time}}
\Large
\begin{enumerate}

\item Objects move in a straight line at a constant speed unless a force acts on them; hoverpuck demo
\item Forces cause their velocities to change size and direction (accelerate); cart demo
\item Gravity is such a force!

\end{enumerate}
}

\frame{\frametitle{\textbf{Newton's first law}}
\BC
\Large
What happens to a ship at sea whose engines die?
\EC
\bigskip
\Large
\bigskip
\bigskip
\bigskip

\color{A}A: It comes to a stop \\
\color{B}B: It keeps going forward until it hits something \\
}

\frame{\frametitle{\textbf{Newton's first law}}
\BC
\Large
What happens to a spacecraft whose engines die?
\EC
\bigskip
\Large
\bigskip
\bigskip
\bigskip

\color{A}A: It comes to a stop \\
\color{B}B: It keeps going forward until it hits something 
}


\frame{\frametitle{\textbf{Newton's first law}}
\BC
\Large
No motive force is required to keep things moving. They do that on their own.

\pause
\bigskip
\bigskip

On Earth things come to rest only because of the forces of friction. (Aristotle was wrong!)

\bigskip
\bigskip

In space we don't have these, so things ``coast'' forever without another force to change their motion.
\EC
}

\frame{\frametitle{\textbf{Newton's biggest idea: the second law of motion}}

\Huge
\BC

``Forces cause objects to accelerate''

\bigskip
\bigskip
\bigskip

$F=ma$

\bigskip
\bigskip
\bigskip

$F/m = a$

\bigskip
\bigskip
\bigskip

\Large

``The strength of a force, divided by the mass of the thing it acts on, gives that thing's acceleration''
\EC
}

\frame{\frametitle{\textbf{Acceleration}}
\Large
Acceleration has a direction; it can increase, decrease, or redirect an object's velocity:

\BI
\item Apply engine power to a car going East: force to the East $\rightarrow$ it goes East faster
\item Apply brakes to a car going East: force to the West $\rightarrow$ it goes East more slowly
\item Turn steering wheel left: force to the North $\rightarrow$ car starts traveling Northeast
\EI
}

\frame{\frametitle{\textbf{The force of gravity}}
\Large
\BC
Newton discovered:

$$
F_{\rm grav} = \frac{G \times ({\rm mass\, of\, object\, A}) \times ({\rm mass\, of\, object\, B})}{({\rm distance\, between\, them})^2}
$$

\pause

or, in mathematical shorthand,

$$
F_{\rm grav} = \frac{Gm_1m_2}{r^2}.
$$

\EC

\large\pause

$G$ is just a number telling how strong gravity is: about a ten-billionth of the weight of an apple for two kilogram objects a meter apart.

}

\frame{
\Large
Suppose two asteroids are floating out in space. Asteroid A is twice as massive as asteroid B.

\bigskip

If the force of A's gravity on B is ten tons, the force of B's gravity on A will be...

\huge

\bigskip

\color{A}A: 5 tons\\
\color{B}B: 10 tons\\
\color{C}C: 20 tons\\
\color{D}D: 40 tons
}


\frame{\frametitle{\textbf{The solution}}

\large

Let's look at that expression again:

$$
F_{\rm grav} = \frac{Gm_1m_2}{r^2}.
$$

(Remember, $r$ is the distance between the objects)

\bigskip
\bigskip

Notice that switching $m_1$ (one asteroid) and $m_2$ (the other) in this expression doesn't affect the result!

\pause
\bigskip
\bigskip
\bigskip

``For every action there is an equal and opposite reaction'': if the Earth's gravity pulls me down with a force of 160 pounds, I pull {\it up on the Earth} with the same force of 160 pounds.
}

\frame{
\Large
Suppose two asteroids are floating out in space, 20 miles apart. Asteroid A is twice as massive as asteroid B,
and the force of A's gravity on B is 20 tons.

\bigskip

Suppose I now move the two asteroids further apart, so they're 40 miles apart. What will the force of A's gravity on B be now?

\huge

\bigskip

\color{A}A: 5 tons\\
\color{B}B: 10 tons\\
\color{C}C: 20 tons\\
\color{D}D: 40 tons\\
\color{E}E: 80 tons
}

\frame{\frametitle{\textbf{The solution}}
\Large
Recall:

$$
F_{\rm grav} = \frac{Gm_1m_2}{\color{Red}r^2}.
$$

\bigskip
\bigskip
\bigskip

If I increase the distance between the asteroids by a factor of 2, then I increase the denominator of this fraction by a factor of \underline{\hspace{4em}},
which will \underline{\hspace{4em}} the whole fraction by a factor of \underline{\hspace{4em}}.
}

\frame{

\BC
\Huge
Complete {\it Lecture Tutorials pp. 29-32.}

\bigskip
\large

We will do something else after this.
\EC
}



\frame{

\Large
Suppose I have two satellites next to each other orbiting the Earth at a distance $r$.
The small one has a mass $m_A$; the big one has mass
$m_B$; the Earth has mass $m_E$.

\bigskip

What is the force of gravity (``weight'') on satellite A from the Earth?

\pause
$$ 
F = \frac{G \times m_E \times m_A}{r^2}
$$
}

\frame{

\Large
Suppose I have two satellites next to each other orbiting the Earth at a distance $r$.
The small one has a mass $m_A$; the big one has mass
$m_B$; the Earth has mass $m_E$.

\bigskip

What is the force of gravity (``weight'') on satellite B from the Earth?

\pause
$$ 
F = \frac{G \times m_E \times m_B}{r^2}
$$

\pause

\color{Red}The {\it force} that they feel depends on their own masses. How does this change how they move?
}

\frame{

\Large
Suppose I have two satellites next to each other orbiting the Earth at a distance $r$.
The small one has a mass $m_A$; the big one has mass
$m_B$; the Earth has mass $m_E$.

\bigskip

What is the acceleration of satellite A because of the Earth's gravity? 

\bigskip

Newton's second law: $F = ma \rightarrow a = F/m$

\bigskip

\pause
$$ 
F = \frac{G \times m_E \times m_A}{r^2}
$$

\pause

$$
a = \frac{G \times m_E \times {\color{B}m_A}}{r^2} \div {\color{B}m_A} = \frac{G \times m_E}{r^2}
$$

\pause

\color{Red}The mass of the satellite cancels from the equation!

}

\frame{

\Large
Suppose I have two satellites next to each other orbiting the Earth at a distance $r$.
The small one has a mass $m_A$; the big one has mass
$m_B$; the Earth has mass $m_E$.

\bigskip

What is the acceleration of satellite B because of the Earth's gravity? 

\bigskip

Newton's second law: $F = ma \rightarrow a = F/m$

\bigskip

\pause
$$ 
F = \frac{G \times m_E \times m_B}{r^2}
$$

\pause

$$
a = \frac{G \times m_E \times {\color{B}m_B}}{r^2} \div {\color{B}m_B} = \frac{G \times m_E}{r^2}
$$

\pause

\color{Red}The mass of the satellite {\it doesn't affect} how Earth's gravity makes it move!

}

\frame{
\large

``Some of you were surprised in class that the time it takes a planet to go around the Sun doesn't depend at all on its mass. Based on your answer to the last question, does this make sense? Explain.''

\bigskip
\bigskip
\bigskip\pause

\it Yes, this makes sense because since $a=F/m$ is an inverse relationship, 
if the force were to double due to the mass increase, 
when it is divided by the new doubled mass, the acceleration will be the same due to the inverse relationship. 
If both are doubled and then divided, 
the acceleration would stay the same...

\rm

\begin{flushright}--Sam
\end{flushright}

\bigskip
\bigskip
\bigskip\pause

\it Force, which equals $ma$, is equal to $GMm/r^2$. The $m$ from both sides cancels.

\rm

\begin{flushright}--Yihong
\end{flushright}


}



\frame{
\large
\color{A}
Student A: ``Kepler's laws say that the planets orbit around the Sun in elliptical orbits, with the Sun fixed at one focus of the ellipse. The Sun doesn't move.

\bigskip
\bigskip
\bigskip
\color{B}
Student B: ``But the laws of gravitation say that if the Sun pulls on the planets, which it must in order to hold them in their orbits, the planets must pull back on the Sun.
This pull makes the Sun accelerate, so it has to wobble a bit.''

\bigskip
\bigskip
\bigskip
\color{white}
Who's right?

\pause

\bigskip
\bigskip
\bigskip
\color{A}
Student A: ``Why don't we see the wobble, then, if the force pulling on the Sun is the same as the force pulling on the planets?''
}

\frame{
\large

Student A: ``Why don't we see the wobble, then, if the force pulling on the Sun is the same as the force pulling on the planets?''

\bigskip
\bigskip
\bigskip

\large

\color{A}A: The gravitational forces from all the planets on the Sun cancel each other out \\ \bigskip 
\color{B}B: The planets are so far away that the force they exert on the Sun is small \\ \bigskip 
\color{C}C: The Sun's mass is so big that this amount of force doesn't affect it that much \\ \bigskip 
\color{D}D: We {\it do} see this wobble, if we look closely enough

\bigskip
\bigskip
\bigskip
\color{white}
Someone else might see it, too...
}

\frame{\frametitle{\textbf{How does this create circular motion?}}
\Large
Without a force, things travel in straight lines at constant speeds (Newton's first law).

\bigskip

It requires a force {\it directed toward the center} to hold something in circular motion.

\bigskip
\bigskip

Let's demonstrate and watch this.}

\frame{\frametitle{\textbf{What about elliptical orbits?}}

\Large
For gravity, the force depends on the distance from the center, as you know. 

\bigskip

The particular mathematics that produces ellipses is beyond the scope of this class. But we can understand the principles!

\bigskip

To make all the simulations for this class, all I did was program $F = \frac{GMm}{r^2}$ and $F=ma$ into
my computer and make it do the math for me! (This is next week's homework in my other class!)

}

\frame{\frametitle{\textbf{How far can we take this?}}
\BC
\Large

\url{https://www.youtube.com/watch?v=W-csPZKAQc8}

\pause
\bigskip
\bigskip
\bigskip

The only difference between this and what I've done on my own computer is the number of objects!
\EC
}







\end{document}
